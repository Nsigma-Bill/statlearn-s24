\documentclass{article}

\def\ParSkip{} 
\input{../../common/ryantibs}
\usepackage[normalem]{ulem}
\usepackage{animate}

\title{Calibration, Scoring, and Blackwell Approachability \\ \smallskip
\large Advanced Topics in Statistical Learning, Spring 2024 \\ \smallskip
Ryan Tibshirani }
\date{}

\begin{document}
\maketitle
\RaggedRight
\vspace{-50pt}

(This is just an outline of what we will cover. Notes from last year's lecture
on scoring and calibration are available on the course website; they have a
slightly different focus/exposition then what we'll pursue this year, but will 
nonetheless be quite relevant as a reference.)  

\bigskip
the many faces of calibration: 

\begin{itemize}
\item forecasting: sequential and probabilistic notations
\item interval coverage generalizes to PIT calibration 
\item interlude: marginal calibration (critiques)
\item conditional calibration (binary and general case)
\item do linear ensembles retain calibration?
\end{itemize}

\bigskip
calibration is not enough---enter scoring rules:

\begin{itemize}
\item let's build up a scoring role from the ground up
\item irreducible uncertainty, sharpness, calibration 
\item proper scores, score divergence, entropy
\item examples: log score, Brier score, CRPS
\item each has different operating characteristics! 
\end{itemize}

\bigskip 
... and along came Blackwell, many decades earlier: 

\begin{itemize}
\item real-valued payoffs: von Neumann's setting
\item vector valued payoffs: Blackwell's setting
\item response satisfiability, halfspace satisfiability
\item Blackwell approachability theorem
\item (quick proof? with credit to M.\ Jordan)
\item recalibration via Blackwell approachability
\end{itemize}

\end{document}