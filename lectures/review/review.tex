\documentclass{article}

\def\ParSkip{} 
\input{../../common/ryantibs}

\title{Review: Stat/ML in a Nutshell \\ \smallskip
\large Advanced Topics in Statistical Learning, Spring 2024 \\ \smallskip
Ryan Tibshirani}
\date{}

\begin{document}
\maketitle
\RaggedRight
\vspace{-50pt}

We will take a brief tour of core concepts in statistics and machine learning. 
Most of this will/should be familiar to you already. Refer to, e.g.,
\citet{wasserman2004all, hastie2009elements} for (much) more details.

\section{Probability: modes of convergence and limit laws}

\def\asto{\overset{\mathrm{as}}{\to}}
\def\pto{\overset{p}{\to}}
\def\dto{\overset{d}{\to}}

Let $X_n$, $n=1,2,3,\dots$ be a sequence of real-valued random variables. 

\begin{itemize}
\item Almost sure convergence: $X_n \asto X$ means that 
  $\P(\lim_{n\to \infty} X_n = X ) = 1$.   

\item Convergence in probability: $X_n \pto X$ means that, for each
  $\epsilon>0$, $\P(|X_n - X| > \epsilon) \to 0$ as $n \to \infty$.  

\item Convergence in distribution: $X_n \dto X$ means that $\P(X_n \leq x) \to 
  \P(X \leq x)$ at all continuity points $x$ of the law of $X$. 

\item Almost sure convergence implies convergence in probability. 

\item Convergence in probability implies convergence in distribution.

\item Convergence in distribution does \emph{not} imply convergence in
  probability (except in the special case that the limiting distribution is a
  constant).   

\item Convergence in distribution is equivalent to $\E[ f(X_n) ] \to \E[ f(X) ]$
  for all bounded, continuous $f$ (this, together with a collection of other
  equivalences, is often called the portmanteau lemma).  

\item Asymptotic probability notation: $X_n = O_p(a_n)$ means that $X_n/a_n$ is
  bounded in probability, i.e., for each $\epsilon>0$, there exists an $M>0$
  such that for sufficiently large $n$,
  \[
  \P\bigg( \bigg|\frac{X_n}{a_n}\bigg| > M \bigg) \leq \epsilon.
  \]
  Meanwhile, $X_n = o_p(a_n)$ means that $X_n/a_n \pto 0$. 

\item Law of large numbers (LLN): if $X_1,X_2,\dots$ are i.i.d.\ with $\mu =
  \E[X_i]$, then for the sample mean 
  \[
  \bar{X}_n = \frac{1}{n} \sum_{i=1}^n X_n,
  \]
  it holds that \smash{$\bar{X}_n \asto \mu$}.

  Note: this implies \smash{$\bar{X}_n \pto \mu$}, but there are some cases
  where latter holds (convergence in probability) but the former does not
  (almost sure convergence), i.e., when the mean is not defined (infinite). 

\item Central limit theorem (CLT): under the same conditions, if additionally a
  second moment exists and we let $\sigma^2 = \Var[X_i]$, then
  \[
  \frac{\sqrt{n}(\bar{X}_n - \mu)}{\sigma} \dto N(0, 1).
  \]

\item Gilvenko-Cantelli theorem: if $X_1,X_2,\dots$ are i.i.d.\ then for the
  empirical distribution function $F_n$ defined as 
  \[
  F_n(x) = \frac{1}{n} \sum_{i=1}^n 1\{X_n \leq x\},
  \]
  it holds that
  \[
  \sup_x \, |F_n(x) - F(x)| \asto 0.
  \]

\item Kolmogorov-Smirnov theorem: under the same conditions, 
  \[
  \sqrt{n} \sup_x \, |F_n(x) - F(x)| \asto \sup_{t \in [0,1]} \, |B(t)|,
  \]
  where $B$ is a Brownian bridge (standard Brownian motion subject to $B(0) =
  0$ and $B(1) = 0$).   
\end{itemize}

\section{Probability: basic concentration inequalities} 

\begin{itemize}
\item Markov's inequality: if $X \geq 0$ and $\mu = \E[X]$, then for any
  $a > 0$, 
  \[
  \P(X \geq a) \leq \frac{\mu}{a}.
  \]
  
\item Chebyshev's inequality: let $\mu = \E[X]$ and $\sigma^2 = \Var[X]$, then 
  for any $t > 0$,  
  \[
  \P(|X - \mu| \geq t) \leq \frac{\sigma^2}{t^2}.
  \]

\item Hoeffding's inequality: if $X_1,\dots,X_n$ are independent and mean zero
  with $a_i \leq X_i \leq b_i$ for each $i$, then for any $t > 0$,
  \[
  \P( \bar{X}_n \geq t) \leq \exp\bigg( \frac{-2nt^2}{
    \frac{1}{n}\sum_{i=1}^n (b_i^2 - a_i^2)} \bigg).
  \]

\item Bernstein's inequality: if $X_1,\dots,X_n$ are independent and mean zero
  with $\sigma_i^2 = \Var[X_i]$, and $|X_i| \leq M$ for each $i$, then for any
  $t > 0$,  
  \[
  \P( \bar{X}_n \geq t) \leq \exp\bigg( \frac{-nt^2/2}{
    \frac{1}{n}\sum_{i=1}^n \sigma_i^2 + Mt/3} \bigg).
  \]
\end{itemize}

\section{Statistics: maximum likelihood and GLMs}

\subsection{Maximum likelihood}

Let $\{P_\theta : \theta \in \Theta\}$ be a parametric family of models, and 
write $p_\theta$ for the probability density function (or probability mass
function, in the discrete case). We write $L(\theta; z) = p_\theta(z)$ for the
likelihood function.

Then given i.i.d.\ $Z_1,\dots, Z_n \sim P_\theta$, the problem
\[ 
\maximize_{\theta \in \Theta} \; \prod_{i=1}^n L(\theta; Z_i),
\] 
is called \emph{maximum likelihood estimation}, and a maximizer
\smash{$\htheta$} of the above is called a \emph{maximum likelihood estimator}
(MLE). (We are often imprecise and call this \emph{the} MLE, ignorning
nonuniqueness of the solution.) Note that we can equivalently write this via the 
negative log-likelihood function $\ell(\theta; z) = - \log L(\theta; z)$, i.e., 
the above is equivalent
to  
\[
\minimize_{\theta \in \Theta} \; \sum_{i=1}^n \ell(\theta; Z_i).
\]
The \emph{Fisher information} is given by the expectation of the Hessian,
\[
I(\theta) = \E[\nabla^2 \ell(\theta; Z) ],
\]
where the expectation is taken over $Z \sim P_\theta$. In the univariate case, 
$\Theta \subseteq \R$, this is simply 
\[
I(\theta) = \E\bigg[ \frac{d^2 \ell(\theta; Z)}{d\theta^2} \bigg].
\]
Under appropriate regularity conditions, when $Z_1,\dots,Z_n \sim P_\theta$, 
\[
\frac{\sqrt{n}(\htheta - \theta)}{s_n} \dto N(0, 1),
\]
where \smash{$s_n = 1/\sqrt{I(\htheta)}$}. In the multivariate case, $\Theta
\subseteq \R^d$, under regularity conditions,  
\[
\sqrt{n} S_n^{-1} (\htheta - \theta) \dto N(0, I_d),
\]
where $S_n = I(\htheta)^{-1/2}$, which is the symmetric square root of the
inverse Fisher information matrix at the MLE.

The following result is also true (under regularity conditions):
\[
\sqrt{n} (\htheta - \theta) \dto N(0, I(\theta) ^{-1}).
\]
From this we can discern
\[
\|\htheta - \theta\|_2^2 = O_p \bigg( \frac{\tr[I(\theta) ^{-1}]}{n} \bigg),    
\] 
and under standard conditions this will be $O_p(d/n)$. This should give you a
rough sense for the squared error of the MLE, in standard (idealized)
settings.\footnote{ Note carefully, however, that for the asymptotic normal
  approximation to be accurate we require $d \ll n$; when $d$ scales
  proportionally to $n$, $d/n \to \gamma > 0$, we get very different asymptotic
  behavior.} 
We will do a related but even simpler and more direct calculation later for
least squares regression. 

\subsection{Generalized linear models}

An \emph{exponential family distribution} is of the form
\[
p_\eta(z) = \exp\big(T(z)^\T \eta - \psi(\eta)\big) h(z),
\]
Here $\eta \in \R^d$ is called the \emph{natural parameter} of the exponential  
family (often we use $\eta$, rather than $\theta$, here). We call the function
$\psi$ the \emph{log-partition function}, we call $T$ the \emph{sufficient
  statistic}, and we call $h$ the \emph{base measure}. 

A remarkable fact: in any exponential family distribution, the log-partition
function $\psi$ is always convex, by virtue of the fact that $p_\eta$ must be a
bona fide density and therefore must integrate to one (sum to one, in the
discrete case). This means that the map $\eta \mapsto p_\eta(z)$ is always
log-concave (for fixed $z$), and the resulting maximum likelihood problem       
\[
\minimize_\eta \; -\frac{1}{n} \bigg( \sum_{i=1}^n T(Z_i) \bigg)^{\hspace{-2pt}
  \T} \eta + \psi(\eta)    
\]
is always a convex optimization problem. (Convexity and optimization will be
discussed shortly.) 

A \emph{generalized linear model} (GLM) builds off the exponential family
distribution. We observe independent draws of a response variable $y_i$,
$i=1,\dots,n$, with each one sampled from an exponential family
distribution. The form here is common (the functions $T, \psi, h$ are common),
but we have sample-specific natural parameters: to each $y_i$, we assign a
natural parameter $\eta_i$, and model it as  
\[
\eta_i = x_i^\T \beta
\] 
for a feature vector $x_i \in \R^d$ and parameter $\beta \in \R^d$. Maximum
likelihood becomes:   
\[
\minimize_\beta \; \sum_{i=1}^n \Big( -T(y_i) x_i^\T \beta + \psi(x_i^\T \beta)
\Big).  
\]
Maximum likelihood in a GLM has many special properties. For example, the
Hessian of the criterion is the same as the Fisher information
matrix,\footnote{Here we are referring to the total Fisher information matrix,
  which has been summed over the observations.} 
since the second-order term has no dependence on $y_i$. 

The three most important special cases are the Gaussian: \smash{$\psi(u) =
  \frac{u^2}{2}$}, Bernoulli: $\psi(u) = \log(1+e^u)$, and Poisson: $\psi(u) =
e^u$ families. This gives rise to the following maximum likelihood problems,
\begin{alignat*}{5}
&\text{Gaussian}: \quad 
&&\minimize_\beta \; \sum_{i=1}^n \bigg(-y_i x_i^\T \beta + 
\frac{(x_i^\T \beta)^2}{2} \bigg) \qquad 
&&\text{(where each $y_i \in \R$)} \\
&\text{Bernoulli}: \quad 
&&\minimize_\beta \; \sum_{i=1}^n \Big( -y_i x_i^\T \beta + 
\log(1 + \exp(x_i^\T \beta)) \Big) \qquad
&&\text{(where each $y_i \in \{0,1\}$)} \\
&\text{Poisson}: \quad
&&\minimize_\beta \; \sum_{i=1}^n \Big( -y_i x_i^\T \beta + 
\exp(x_i^\T \beta) \Big) \qquad
&&\text{(where each $y_i \in \N$)}. 
\end{alignat*}
These are known as \emph{least squares regression}, \emph{logistic regression},
and \emph{Poisson regression}, respectively. The Hessians in these problems are:  
\begin{alignat*}{5}
&\text{Gaussian}: \quad
&&H(\beta) = \sum_{i=1}^n x_i x_i^\T \qquad
&& \text{(no dependence on $\beta$)} \\
&\text{Bernoulli}: \quad 
&&H(\beta) = \sum_{i=1}^n x_i p(x_i^\T \beta) (1 - p(x_i^\T \beta)) x_i^\T 
\qquad    
&&\text{where $p(u) = 1/(1+e^{-u})$} \\
&\text{Poisson}: \quad
&&H(\beta) = \sum_{i=1}^n x_i \lambda(x_i^\T \beta) x_i^\T \qquad
&&\text{where $\lambda(u) = e^u$}.
\end{alignat*}
This can be written more succintly in terms of $X \in \R^{n \times d}$, the
predictor matrix (whose $i\th$ row is $x_i$): 
\begin{alignat*}{5}
&\text{Gaussian}: \quad
&&H(\beta) =X^\T X \qquad
&& \text{(no dependence on $\beta$)} \\
&\text{Bernoulli}: \quad 
&&H(\beta) = X^\T W(\beta) X \qquad    
&&\text{where $W(\beta)$ is diagonal with $[W(\beta)]_{ii} = p(x_i^\T \beta)
  (1 - p(x_i^\T \beta))$} \\
&\text{Poisson}: \quad
&&H(\beta) = X^\T W(\beta) X \qquad
&&\text{where $W(\beta)$ is diagonal with $[W(\beta)]_{ii} = \lambda(x_i^\T
  \beta)$}. 
\end{alignat*}

\section{Convexity and optimization}

A \emph{convex set} $C \subseteq \R^d$ is one that satisfies 
\[
x, y \in C \implies t x + (1-t) y \in C, \quad \text{for all $t \in [0,1]$}. 
\]
A \emph{convex function} $f : \R^d \to (-\infty, \infty]$ is one such that its
effective domain $\dom(f)$ (the set of $x$ for which $f$ is defined and finite)
is a convex set, and 
\[
f \big(t x + (1-t) y\big) \leq t f(x) + (1-t) f(y),  \quad \text{for all $x,y
  \in \dom(f)$ and $t \in [0,1]$}. 
\]
It is called \emph{strictly convex} if the above inequality is satisfied
strictly whenever $x \not= y$ and $t \in (0,1)$. Concavity of $f$ means that
$-f$ is convex, and same with strict concavity. 

A differentiable function $f$ is convex if and only if $\dom(f)$ is convex and  
\[
f(y) \geq f(x) + \nabla f(x)^\T (y-x), \quad \text{for all $x,y \in \dom(f)$}, 
\]
where $\nabla f$ denotes the gradient of $f$. This is called the
\emph{first-order characterization} of convexity. A twice differentiable
function $f$ is convex if and only if $\dom(f)$ is convex and
\[
\nabla^2 f(x) \succeq 0, \quad \text{for all $x \in \dom(f)$}.
\]
where $\nabla^2 f$ denotes the Hessian of $f$, and we write $A \succeq 0$ to
mean that a matrix $A$ is positive semidefinite (symmetric and its smallest
eigenvalue is nonnegative). This is called the \emph{second-order
  characterization} of convexity. 

From the expressions for the Hessians given above, you can check directly that
the criterions in the least squares, logistic regression, and Poisson regression
problems are convex.  

\subsection{Optimization basics}

An \emph{optimization problem} is of the form
\begin{alignat*}{2}
&\minimize_x \quad && f(x) \\
&\st \quad && g_i(x) \leq 0, \; i=1,\dots,m \\ 
& && h_j(x) = 0, \; j=1,\dots,k.
\end{alignat*}
Here the minimization is implicitly restricted to \smash{$D = \dom(f) \cap 
  \bigcap_{i=1}^m \dom(g_i) \cap \bigcap_{j=1}^k \dom(h_j)$}, the intersection
of relevant effective domains.

A \emph{convex optimization problem} is one of the above form such that $f$ and
$g_i$, $i=1,\dots,m$ are all convex functions, and $h_j$, $j=1,\dots,k$ are
all affine functions. In other words, the problem 
\begin{alignat*}{2}
&\minimize_x \quad && f(x) \\
&\st \quad && g_i(x) \leq 0, \; i=1,\dots,m \\ 
& && Ax = b,
\end{alignat*}
is convex whenever $f$ and $g_i$, $i=1,\dots,m$ are convex (and $A$ and $b$ are
arbitrary). 

The function $f$ in such problems is called the \emph{objective} or
\emph{criterion}. A \emph{feasible point} is a point in $D$ such that all
constraints (inequality and equality constraints) are met. The infimal criterion
value among all feasible points is often denoted $f^\star$, and called the
\emph{optimal value}. 

A feasible point that achieves the optimal value is denoted $x^\star$ (note that
$f^\star = f(x^\star)$), and is called a  \emph{solution} or
\emph{minimizer}. However, in statistics, we often use ``hat  notation'', as in
\smash{$\htheta$} for the MLE. An important fact: if the criterion $f$ is
strictly convex, and a solution exists, then it must be unique.

A point \smash{$\bar{x}$} is called a \emph{local solution} if it is feasible
and there is some $\delta>0$ such that 
\[
f(\bar{x}) \leq f(x), \quad \text{for all feasible $x$ such that
$\|x - \bar{x}\|_2 \leq \delta$}.
\]
For a convex optimization problem, the following holds: any local solution
\smash{$\bar{x}$} must also be a global solution: \smash{$f(\bar{x}) \leq f(x)$}
for all feasible points $x$. This result is so important that it may as well be
called the \emph{fundamental theorem of convex optimization}.   

A huge number of estimators in statistics and machine learning are defined by
optimization problems, and many of these are convex problems. There has been a
surge in interest in (smooth) nonconvex optimization recently due to the rise of
deep neural networks.  

\subsection{Subgradients}

For a function $f$ on $\R^d$, we say that $s \in \R^d$ is a \emph{subgradient} 
of $f$ at $x \in \dom(f)$ provided that
\[
f(y) \geq f(x) + s^\T (y-x), \quad \text{for all $y \in \dom(f)$}.
\]
This is analogous to the first-order characterization for convexity, where $s$
playes the role of $\nabla f(x)$. We write $\partial f(x)$ for the set of all
subgradients of $f$ at $x$, which is called the \emph{subdifferential}.

For convex $f$, if $f$ is differentiable at $x$, then $\partial f(x) =
\{x\}$. The converse is true as well (if the subdifferential is a singleton,
then $f$ must be differentiable at $x$, with its gradient given by the single
subgradient). 

Subgradients play a big role in nonsmooth optimization, but are also key for 
statistical analysis for certain problems/estimators, such as the lasso. For
example, the  \emph{subgradient optimality condition} gives us the following 
characterization:
\[
\text{$x$ minimizes $f$} \iff 0 \in \partial f(x).
\]
We note that this is true without any assumptions on $f$ (no need to assume
convexity of $f$). The proof is so simple it's nearly vacuous: $x$ minimizes $f$
if and only if $f(y) \geq f(x)$ for all $y$, which is the definition of 0 being
a subgradient of $f$ at $x$. And for differentiable convex $f$, we get the
familiar condition $0 = \nabla f(x)$.

\subsection{Algorithms}

The two most basic methods for unconstrained minimization of a function $f$ are
\emph{gradient descent}:  
\[
x_{k+1} = x_k - t_k \nabla f(x_k), \quad k=1,2,3,\dots,
\]
for differentiable $f$, where each $t_k \geq 0$ is a step size, and
\emph{Newton's method}: 
\[
x_{k+1} = x_k - t_k [\nabla^2 f(x_k)]^{-1} \nabla f(x_k), \quad k=1,2,3,\dots, 
\]
for twice differentiable $f$, where again each $t_k \geq 0$ is a step size. For
nonsmooth and/or constrained optimization, variants exist (as do many, many
other optimization algorithms). 

These two algorithms (or close cousins of them) appear frequently in statistics
and machine learning. For example, for optimization in a GLM, Newton's method 
reduces to what is known as \emph{iteratively reweighted least squares} (IRLS), 
which is important both computationally and inferentially. 

\section{Regression}

Suppose that we observe $(X,Y)$ from some unknown joint distribution, where $Y
\in \R$, and we are interested in predicting $Y$ from $X$. Over all functions
$f$, the prediction error as measured in terms of squared loss        
\[
\E\big[ (Y - f(X))^2\big]
\]
is minimized at
\[
f(x) = \E[Y | X=x], 
\]  
called the \emph{regression function} of $Y$ on $X$. If we observe i.i.d.\ pairs 
$(x_i,y_i)$, $i=1,\dots,n$ from the same joint distribution as $(X,Y)$, then we
can use this to estimate $f$. This is the most generic problem formulation 
available for regression, and we'll call it the \emph{XY-Pairs model}.  

\subsection{Signal-plus-noise models}

You'll often see authors writing down a model of the form:   
\[
y_i = f(x_i) + \epsilon_i, \quad i=1,\dots,n.
\]
This is often referred to as a ``signal-plus-noise'' model. You might initially
think that this is assuming more than an XY-Pairs model, but it's actually the
same, for i.i.d.\ mean zero stochastic errors $\epsilon_i$, $i=1,\dots,n$,
provided we treat the predictors $x_i$, $i=1,\dots,n$ as suitably random as 
well. Let's ammend the above to emphasize this:   
\begin{equation}
\begin{gathered}
\label{eq:random_x}
\text{$(x_i, y_i)$, $i=1,\dots,n$ are i.i.d.}, \\
\text{where each $y_i = f(x_i) + \epsilon_i$, and $\E[\epsilon_i] = 0$}. 
\end{gathered}
\end{equation}
which we'll call the \emph{Random-X signal-plus-noise model}. To see that 
\[
\text{XY-Pairs} \iff \text{Random-X signal-plus-noise},
\]
the key is that we can always define $\epsilon_i = y_i - f(x_i)$. This has zero
mean as $\E[f(x_i)] = \E[\E[y_i|x_i]] = \E[y_i]$.   

But what if we assume that each $x_i$ is fixed? To contrast, we'll call this the
\emph{Fixed-X signal-plus-noise model}: 
\begin{equation}
\begin{gathered}
\label{eq:fixed_x}
\text{$x_i$, $i=1,\dots,n$ are fixed}, \\
\text{$\delta_i$, $i=1,\dots,n$ are i.i.d.}, \\
\text{where each $y_i = f_0(x_i) + \delta_i$, and $\E[\delta_i] = 0$}. 
\end{gathered}
\end{equation}
The interpretation you'll often hear: this is just given by the Random-X 
signal-plus-noise model \eqref{eq:random_x} (which is itself equivalent to the
XY-Pairs model) after we condition on each $x_i$.          

However, there is an important subtlety here! The precise connection is this: 
\begin{gather*}
\text{Fixed-X signal-plus-noise} \iff \text{Conditioning on each $x_i$ in 
  Random-X signal-plus-noise}, \\ 
\hspace{130pt} \text{\emph{provided that each $x_i \indep \epsilon$}}.   
\end{gather*}
The reason for the extra independence assumption is this: if didn't have
independence of each $x_i$ and $\epsilon_i$, then after conditioning on $x_i$,
$i=1,\dots,n$ in \eqref{eq:random_x}, the distributions of $\delta_i =
\epsilon_i | x_i$, $i=1,\dots,n$ need not be identical.  

\subsection{Independence of $x_i$ and $\epsilon_i$}

To be frank, assuming independence of $x_i$ and $\epsilon_i$ is fairly common.
But to be clear, it is an \emph{extra assumption} and does not fall out of the
generic XY-Pairs model. It's also not totally harmless. To see this, suppose
that we were in a situation where the true model is:  
\begin{gather*}
\text{$(x_i, z_i, y_i)$, $i=1,\dots,n$ are i.i.d.}, \\
\text{where each $y_i = f(x_i) + g_0(z_i) + \xi_i$, and $\E[\xi_i] = 0$}, 
\end{gather*}
Let's even assume that each $x_i \indep \xi_i$, and $\E[g_0(z_i) | x_i] =
0$. Then the regression function (of $y_i$ on $x_i$) is indeed $f$. But in the  
signal-plus-noise model for $y_i$ on $x_i$, the noise variable is 
\[
\epsilon_i = g_0(z_i) + \xi_i,
\]
and if $z_i$ is dependent on $x_i$, then $\epsilon_i$ will be too.

Therefore, assuming that $\epsilon_i$ is independent of $x_i$, in general, is
like assuming that \emph{any omitted variables are independent of the current
  ones}. Which could certainly be seen as a strong assumption.   

\subsection{Prediction error metrics}

\def\Err{\mathrm{Err}}
\def\Risk{\mathrm{Risk}}

The Random-X and Fixed-X settings call to mind similar but distinct notions of 
prediction error. In the former, it is natural to ask: how would well would we
predict at a new test point $(x_0,y_0)$? That is, given an estimator
\smash{$\hf$} that we fit to training data $(x_i,y_i)$, $i=1,\dots,n$, we 
consider       
\begin{equation}
\label{eq:random_x_error}
\Err(\hf) = \E\big[ (y_0 - \hf(x_0))^2 \big],
\end{equation}
where the expectation is taken over the training data $(x_i,y_i)$, $i=1,\dots,n$
and the test point $(x_0,y_0)$, all i.i.d. We'll call \smash{$\Err(\hf)$} the
\emph{out-of-sample prediction error}, often (here and frequently with other
terms as well) dropping ``prediction'' for simplicity.  

Meanwhile, in the Fixed-X setting, it would instead be more natural to consider       
\begin{equation}
\label{eq:fixed_x_error}
\Err(\hf; x_{1:n}) = \E\bigg[ \frac{1}{n}\sum_{i=1}^n (y_i^* - \hf(x_i))^2
\bigg], 
\end{equation}
where the expectation is taken each over $y_i$ and an independent copy $y_i^*$
of the response value. The notation \smash{$\Err(\hf; x_{1:n})$} emphasizes the 
dependence of this metric, which we call the \emph{in-sample prediction
  error},\footnote{Don't confuse this with training error! This may be tempting
  as the name ``in-sample'' may remind you of training error. But they're not
  the same thing! (Even worse, some other authors use ``in-sample error'' and
  ``training error'' synonomously).}     
on $x_{1:n} = \{x_1,\dots,x_n\}$. It can be interpreted as the prediction error
\emph{had we encountered new responses at the exact same feature values}.

If you think the latter interpretation sounds strange, then---well, you're
right, it is a bit strange in a modern prediction context! (More discussion
soon.)    

\subsubsection{Risk equivalents} 

Assuming independence of $\epsilon_0 = y_0 - f(x_0)$ and $x_0$, we have, by
adding and subtracting $f(x_0)$ in \eqref{eq:random_x_error} and expanding:  
\[
\underbrace{\E\big[ (y_0 - \hf(x_0))^2 \big]}_{\Err(\hf)} =
\underbrace{\E\big[ (f(x_0) - \hf(x_0))^2 \big]}_{\Risk(\hf)} +\, \sigma^2,
\]
where $\sigma^2 = \Var[\epsilon_0]$ is the noise variance (also known as the 
\emph{Bayes error} or \emph{irreducible error} in the current context). We'll
call the first term on the right-hand side the \emph{out-of-sample prediction
  risk}\footnote{Admittedly, many authors use the terms ``risk'' and ``error'' 
  interchangeably, and do not use it to distinguish whether the target is
  $f(x_0)$ or $y_0$. In fact, we will do so ourselves in future lectures!
  However, when considering both metrics in a single discussion---either 
  \eqref{eq:random_x_error} and \eqref{eq:random_x_risk}, or 
  \eqref{eq:fixed_x_error} and \eqref{eq:fixed_x_risk}---it is helpful to use  
  nomenclature to distinguish the two.} 
and denote it by \smash{$\Risk(\hf)$}. Like the above relationship, we can also
write down a similar equivalence to risk in the Fixed-X context, by adding and
subtracting $f(x_i)$ in each summand of \eqref{eq:fixed_x_error} and expanding:  
\[
\underbrace{\E\bigg[ \frac{1}{n}\sum_{i=1}^n (y_i^* - \hf(x_i))^2 
  \bigg]}_{\Err(\hf; x_{1:n})} =   
\underbrace{\E\bigg[ \frac{1}{n}\sum_{i=1}^n (f(x_i) - \hf(x_i))^2
  \bigg]}_{\Risk(\hf;  x_{1:n})} +\, \sigma^2, 
\]
where $\sigma^2 = \Var[\epsilon_i]$ is the noise variance. Similarly, we'll call
the first term on the right-hand side the \emph{in-sample prediction risk}, and
denote it by \smash{$\Risk(\hf; x_{1:n})$}. 

So to recap, in the Random-X setting, we have as a natural metric  
\begin{equation}
\label{eq:random_x_risk}
\Risk(\hf) = \E\big[ (f(x_0) - \hf(x_0))^2 \big],
\end{equation}
called the out-of-sample risk, which only differs from the out-of-sample error
by $\sigma^2$. Meanwhile, in the Fixed-X setting, we have the natural metric 
\begin{equation}
\label{eq:fixed_x_risk}
\Risk(\hf; x_{1:n}) = \E\bigg[ \frac{1}{n}\sum_{i=1}^n (f(x_i) - \hf(x_i))^2
\bigg], 
\end{equation}
called the in-sample risk, which again only difference from the in-sample error
by $\sigma^2$.

\subsubsection{Interlude: same roses, different names}

Note that we can also view \smash{$\Risk(\hf)$} and \smash{$\Risk(\hf;
  x_{1:n})$} in terms of the $L^2(P)$ and $L^2(P_n)$ error metrics:
\begin{align*}
\Risk(\hf) &= \E \big\|f - \hf\big\|_{L^2(P)}^2 
= \E\bigg[ \int (f(x) - \hf(x))^2 \, dP(x) \bigg], \\
\Risk(\hf; x_{1:n}) &= \E \big\|f - \hf\big\|_{L^2(P_n)}^2  
= \E\bigg[ \int (f(x) - \hf(x))^2 \, dP_n(x) \bigg],
\end{align*}
where $P$ is the feature distribution and $P_n$ the empirical distribution of
$x_{1:n}$. Thus out-of-sample and in-sample error could also be called the
(expected) \emph{$L^2(P)$ risk} and \emph{$L^2(P_n)$ risk},
respectively. Other names you'll hear are the \emph{population $L^2$ risk} and
\emph{empirical $L^2$ risk}, respectively.    

Empirical process theory gives us bounds on the $L^2(P_n)$ and $L^2(P)$ norms
(typically, these bounds hold for all functions in some particular function
class, with high probability over draws of $x_1,\dots,x_n$). We'll cover this
later in the course.    

\subsubsection{Discussion: how different are they?}

Classically, statisticians really love in-sample risk \eqref{eq:fixed_x_risk}. A
big part of this is probably that it gives a very clean framework in which we
can develop various powerful tools for analysis (e.g., the covariance formula,
Stein's unbiased risk estimator, etc.). To be more charitable, there may have
also been more genuine motivations for in-sample risk from experimental design:
if the values $x_1,\dots,x_n$ were chosen (i.e., designed), and we ran an
experiment in which we measured $y_1,\dots,y_n$, then looking at
\smash{$\Err(\hf; x_{1:n})$} would not seem so crazy.   

In machine learning and in most modern application of statistical prediction,
however, out-of-sample risk \eqref{eq:random_x_risk} seems to be much more 
fitting. So this begs the question: how different can \eqref{eq:random_x_risk}  
and \eqref{eq:fixed_x_risk} be? This turns out to be a rich question, and there
is a lot to say about it. The short answer is:
\begin{itemize}
\item \emph{usually not very different} in classical regimes (low-dimensional,
  smooth \smash{$\hf,f$}); but  
\item \emph{can be very different} in modern regimes (high-dimensional, and/or
  nonsmooth \smash{$\hf,f$}).     
\end{itemize}
For example, in the interpolation regime (of great interest recently, along with
overparametrized machine learning methods more generally), they can be extremely
different. Again, we may study this later in the course if there is time and 
interest. For now, we'll just look at least squares regression which already
provides some insights into the matter. Before that, we'll quickly review the 
bias-variance decomposition. 

\subsection{Bias and variance} 

\def\Bias{\mathrm{Bias}}

In general, for an estimator \smash{$\htheta$} of $\theta$, it holds (just add
and subtract \smash{$\E[\htheta]$}, and expand) that
\[
\E[(\theta - \htheta)^2] = 
\underbrace{(\theta - \E[\htheta])^2}_{\Bias^2(\htheta)} \,+\,
\underbrace{\E[(\htheta - \E[\htheta])^2]}_{\Var(\htheta)}. 
\]
This is commonly known as the \emph{bias-variance decomposition}. In a Fixed-X
regression context, where the feature values $x_{1:n} = \{x_1,\dots,x_n\}$ are
treated as fixed, we just apply this to each \smash{$\htheta = \hf(x_i)$} and
average over $i=1,\dots,n$ to obtain 
\[
\Risk(\hf; x_{1:n}) = \frac{1}{n} \sum_{i=1}^n \Bias^2(\hf(x_i)) + \frac{1}{n}
\sum_{i=1}^n \Var(\hf(x_i)). 
\] 
Meanwhile, in a Random-X setting, we can condition on the test point $x_0$,
apply this to \smash{$\htheta = \hf(x_0)$}, and then integrate over $x_0$ to
obtain   
\[
\Risk(\hf) = \E\big[\Bias^2(\hf(x_0) | x_0)\big] + \E\big[ \Var(\hf(x_0) | x_0)
\big],
\] 
where the expectation is over $x_0$. 

The bias-variance decomposition is useful for a variety of reasons, from
analytical to conceptual. The typical trend is that underfitting means high bias
and low variance, whereas overfitting means low bias but high variance. And the
conventional wisdom is that we want to balance these in order to make accurate
predictions.

Interestingly, the bias-variance decomposition has been called into question in
recent years, with regards to the study of overparametrized estimators in
machine learning. But I don't really think that the fundamental idea of 
``balancing bias and variance being a good thing'' is actually contradicted 
here. It's more that bias and variance can manifest themselves in strange ways
in these settings. We'll cover this later in the course. 

Finally, it should be noted that the decompositions presented above aren't the
only bias-variance decompositions available, and certainly not the only ones
that are useful. We can obtain other ones by conditioning on---and later
integrating out over---other parts of data. For example, in the next subsection,
to analyze the risk of least squares, we will see that it is useful to also
condition on the training features.

\subsection{Least squares regression}

Let $X \in \R^{n \times d}$ denote the predictor matrix (whose $i\th$ row is
$x_i$). Analogously, let $Y \in \R^n$ denote the vector of response
variables. \emph{Least squares regression} of $Y$ on $X$ is given by 
\[
\hf(x) = x^\T \hbeta,
\]
where 
\[
\hbeta = \argmin_\beta \; \|Y - X\beta\|_2^2 = (X^\T X)^{-1} X^\T Y.  
\]
Critically, here we are assuming that $\rank(X) = d$, which necessarily requires
that $d \leq n$, which makes the least squares criterion admit a unique
solution. Note that the fitted values (i.e., in-sample predictions) are 
\[
X\hbeta = X (X^\T X)^{-1} X^\T Y = P_X Y,
\]
where $P_X = X (X^\T X)^{-1} X^\T $ denotes the projection onto the column space
of $X$. 

\subsubsection{In-sample risk}

To investigate its risk properties with as simple math as possible, let's assume
an underlying linear model  
\begin{equation}
\label{eq:linear_model}
Y = X\beta_0 + \epsilon,
\end{equation}
with $X$ fixed, and $\epsilon \in \R^n$ having i.i.d.\ entires with mean zero
and variance $\sigma^2$. First, observe that   
\[
\E[X \hbeta] = P_X \E[Y] = P_X X \beta_0 = X \beta_0,
\]
so least squares regression has zero bias, in the in-sample sense. Thus its
in-sample error is pure variance, and we compute 
\begin{equation}
\label{eq:least_squares_in}
\Risk(\hf; x_{1:n}) = \frac{1}{n} \tr[\Var(P_X Y)] = \sigma^2 \frac{d}{n}.
\end{equation}

\subsubsection{Out-of-sample risk}

Meanwhile, for the out-of-sample risk we'll take $X$ to be random, and assume
$X$ is independent of $\epsilon$ in \eqref{eq:linear_model}, with $(x_0,y_0)$
being another i.i.d.\ draw from the same linear model. Then, conditioning on
both $X,x_0$, we can see that 
\[
\E [x_0^\T \hbeta \,|\, X,x_0] = x_0^\T (X^\T X)^{-1} X^\T \E[Y|X] = x_0^\T \beta_0.   
\]
Hence the out-of-sample bias will still be zero, after integrating over
$X,x_0$. (Note that unbiasedness here doesn't actually require the strong
assumption of $X,\epsilon$ being independent.) To compute the out-of-sample 
variance, we again first condition on $X,x_0$:
\[
\Var(x_0^\T \hbeta \,|\, X, x_0) = x_0^\T (X^\T X)^{-1} X^\T \Var(Y|X)
X (X^\T X)^{-1} x_0 = \sigma^2 x_0^\T (X^\T X)^{-1} x_0.
\]
Then integrating over $X,x_0$ gives the out-of-sample risk:
\begin{equation}
\label{eq:least_squares_out}
\Risk(\hf) = \sigma^2 \tr\Big(\E [x_0 x_0^\T] \, \E [(X^\T X)^{-1}] \Big), 
\end{equation}
where we have used the independence of $X,x_0$. An exact formula will not be
possible in full generality here, since as we can see the out-of-sample risk
depends on the distribution of the predictors. Contrast this with the in-sample
risk, which did not.

However, we can still go further than \eqref{eq:least_squares_out} from several
perspectives.   

\begin{itemize}
\item In the special case when each $x_i \sim N(0,\Sigma)$, with $\Sigma$
  invertible, we can compute the out-of-sample risk exactly. In this case, it
  holds that $X^\T X \sim W(\Sigma, n)$, a Wishart distribution, and $(X^\T
  X)^{-1} \sim W^{-1}(\Sigma^{-1}, n)$, an inverse Wishart distribution. Thus   
  \[
  \Risk(\hf) = \sigma^2 \tr\bigg(\Sigma \frac{\Sigma^{-1}}{n-d-1}\bigg)
  = \sigma^2 \frac{d}{n-d-1}.
  \]

\item If we assume conditions that are standard in random matrix theory (which
  permit Gaussian features but also many other types of feature distributions),
  and let $d/n = \gamma \in (0,1)$, then as $d,n$ grow, 
  \[
  \Risk(\hf) \approx \sigma^2 \frac{\gamma}{1-\gamma}.
  \]
  This agrees with the normal calculation from the last bullet point. Comparing
  the above display to \eqref{eq:least_squares_in}, which we can simply write as 
  \smash{$\Risk(\hf; x_{1:n}) = \sigma^2 \gamma$}, we see that the in-sample and
  out-of-sample risks can be very different---with the latter being much
  larger---when $\gamma$ is close to 1 ($d$ is close to $n$). We'll revisit this 
  and related calculations later in the course.      

\item In general, we can always rewrite \eqref{eq:least_squares_out} as  
  \[
  \Risk(\hf) = \frac{\sigma^2}{n} \tr\Big(\E [X^\T X] \, \E [(X^\T X)^{-1}]
  \Big),  
  \]
  just by using the fact that $x_0$ and the rows of $X$ have the same
  distribution. It can be shown \citep{groves1969note} that 
  \[
  \E[(X^\T X)^{-1}] - (\E[X^\T X])^{-1} \; \text{is positive semidefinite},
  \]
  for any feature distribution such that $X^\T X$ is almost surely
  invertible. Then from the above display,
  \[
  \Risk(\hf) \geq \frac{\sigma^2}{n} \tr\Big(\E [X^\T X] \, (\E[X^\T X])^{-1}
  \Big) = \sigma^2 \frac{d}{n}.
  \]
  That is, the out-of-sample risk is always larger than the in-sample risk for
  least squares.

  On the homework, you will generalize this last result by showing that it still 
  holds without assuming a true linear model (i.e., allowing the regression
  function to be possibly nonlinear). 
\end{itemize}

\subsection{Regularization}

As we just saw, the risk of least squares regression degrades as $d$ grows close
to $n$---in the case of Gaussian features or ``RMT features'' (which we use to
mean feature models compatible with the standard assumptions in random matrix
theory), the out-of-sample risk actually diverges when $d = n$. Meanwhile, the 
least squares estimator is not even well-defined when $d > n$, in that the
optimization problem  
\[
\minimize_\beta \; \|Y - X\beta\|_2^2 
\] 
does not have a unique solution.  

How do we deal with such issues? The short answer is \emph{regularization}. At
its core, regularization provides us with a way to navigate the bias-variance
tradeoff: we (ideally greatly) reduce the variance at the expense of introducing
bias. In regression, canonical choices of regularizers are the $\ell_0$,
$\ell_1$, and $\ell_2$ norms:    
\[
\|\beta\|_0 = \sum_{j=1}^d 1\{\beta_j \not= 0\}, \;\;\;
\|\beta\|_1 = \sum_{j=1}^d |\beta_j|, \;\;\;
\|\beta\|_2 = \bigg(\sum_{j=1}^d \beta_j^2 \bigg)^{1/2}.
\]
This gives rise to the regularized least squares problems: 
\begin{alignat}{3}
\label{eq:subset}
&\text{Best subset selection}: \quad
&&\minimize_\beta \; \|Y - X\beta\|_2^2 + \lambda \|\beta\|_0 \\ 
\label{eq:lasso}
&\text{Lasso regression}: \quad
&&\minimize_\beta \; \|Y - X\beta\|_2^2 + \lambda \|\beta\|_1 \\ 
\label{eq:ridge} 
&\text{Ridge regression}: \quad
&&\minimize_\beta \; \|Y - X\beta\|_2^2 + \lambda \|\beta\|_2^2,
\end{alignat}
where $\lambda \geq 0$ is called the tuning parameter (typically chosen by
cross-validation or similar techniques.)  

\paragraph{One of these is not like the other: convexity.}

Calling $\|\cdot\|_0$ the ``$\ell_0$ norm'' is a misnomer, since it is not a
norm. Critically, $\|\cdot\|_0$ is \emph{not convex}, while $\|\cdot\|_1$ and
$\|\cdot\|_2$ are (indeed, any norm is a convex function). This makes
\eqref{eq:subset} a nonconvex problem, and one that is generally very hard to
solve in practice except for very small $d$ (e.g., its constrained form is known
to be NP-hard). On the other hand, problems \eqref{eq:lasso} and
\eqref{eq:ridge} are convex optimization problems, and many efficient
algorithms exist for them.   

\paragraph{One of these is not like the other: sparsity.}

Meanwhile, best subset selection and the lasso have a special and useful
property: their solutions \smash{$\hbeta$} are \emph{sparse}, which means
\smash{$\hbeta_j = 0$}, for many $j$. Larger $\lambda$ typically means sparser
solutions. This is not true of the ridge regression estimator, which will be
generically dense (all nonzero components), for any $\lambda \geq 0$. Sparsity
is often desirable because it corresponds to performing variable selection in
the fitted linear model.

\section{Classification}

\def\hC{\hat{C}}
\def\hh{\hat{h}}

We conclude with a quick recap of classification. Many of the same ideas from
regression carry over. For $(X,Y) \sim P$, with $Y \in \{0,1\}$, the regression
function is 
\[
f(x) = \E[Y | X=x] = \P(Y=1 | X=x),
\]
which now becomes the conditional probability of observing class 1, given
$X=x$. Over all classifiers $C$, the one that minimizes misclassification risk 
\[
\Risk(C) = \P(Y \not= C(X))
\]
is called the \emph{Bayes classifier}, defined by 
\[
C(x) = 
\begin{cases}
0 & \text{if $f(x) \leq 1/2$} \\
1 & \text{if $f(x) > 1/2$}
\end{cases}.
\]
To see this, consider any classifier $C'$ and any fixed $x$,
\begin{align*}
\P(Y \not= C'(X) \,|\, X=x) 
&= 1 - \P(Y=1, C'(X)=1 \,|\, X=x) - \P(Y=0, C'(X)=0 \,|\, X=x) \\ 
&= 1 - C'(x) f(x) + (1 - C'(x))(1 - f(x)) \\
&= f(x) + (1 - 2f(x))C'(x).
\end{align*}
Thus, we can compute the conditional risk difference between $C'$ and $C$ as
\[
\P(Y \not= C'(X) \,|\, X=x) - \P(Y \not= C(X) \,|\, X=x) 
= (2f(x)-1)(C(x)-C'(x)).
\]
When $f(x)>1/2$, we have $C(x)=1$ by construction, and so the right-hand side
above is nonnegative. When $f(x) \leq 1/2$, we have $C(x)=0$ by construction,
and so again the right-hand side is nonnegative. Therefore we have shown
$\P(Y \not= C'(X) | X=x) - \P((Y \not= C(X) | X=x) \geq 0$ for every $x$;
integrating over $x$ gives the result 
\[
\P(Y \not= C(X)) \leq \P(Y \not= C'(X)),
\]
for any classifier $C'$. 

Many classifiers are \emph{plug-in classifiers}, of the form 
\[
\hC(x) = 
\begin{cases}
0 & \text{if $\hf(x) \leq 1/2$} \\
1 & \text{if $\hf(x) > 1/2$}
\end{cases},
\]
for an estimator \smash{$\hf$} of the regression function. It is often useful to 
recode so that $Y \in \{-1,1\}$, because then plug-in classifiers take the form  
\[
\hC(x) = \sign(\hh(x)),
\]
for some \smash{$\hh$}. 

Note that in this coding, we can write misclassification risk as 
\[
\Risk(\hC) = \P(Y \not= \hC(X)) = \P( Y \hh(X) < 0).
\]
This is of the form 
\[
\E[L(Y \hh(X))], \quad \text{where $L(u) = 1\{u < 0\}$},
\]
where $L$ is clearly a nonconvex function. Often, we replace this with a convex
surrogate loss, such as 
\begin{alignat*}{3}
&\text{Logistic}: \quad 
&& L(u) = \log(1 + \exp(-u)) \\
&\text{Exponential}: \quad 
&&L(u) = \exp(-u) \\
&\text{Hinge}: \quad 
&&L(u) = (1 - u)_+,
\end{alignat*}
as used in logistic regression, AdaBoost, and support vector machines,
respectively. 

\bibliographystyle{plainnat}
\bibliography{../../common/ryantibs}

\end{document}
