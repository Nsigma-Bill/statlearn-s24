\documentclass{article}

\def\ParSkip{} 
\input{../../common/ryantibs}

\title{Minimax Theory for Nonparametric Regression \\ \smallskip
\large Advanced Topics in Statistical Learning, Spring 2024 \\ \smallskip
Ryan Tibshirani}
\author{}
\date{}

\begin{document}
\maketitle
\RaggedRight
\vspace{-50pt}

\section{Introduction}

When we are doing theory for nonparametric regression (or really statistical
estimation in general), how can we tell if a convergence rate that we can prove
for a particular method is ``impressive''? Can the analysis be tightened? Or the
method itself improved? And even if we carried this out, will such refinements 
actually lead to a better convergence rate?   

The answer to the last question can be provided by minimax theory, which is a
set of techniques for characterizing the best worst-case behavior of a procedure
over a class of distributions for a particular statistical learning task. 

Let $\cP$ be a set of distributions, and let $Z_1,\dots,Z_n$ be i.i.d.\ from $P
\in \cP$. Let $\theta(P)$ be some functional of $P$ (we will give several
concrete examples shortly), and let \smash{$\htheta = \htheta(Z_1,\dots,Z_n)$}
denote an estimator of $\theta$, based on the sample $Z_1,\dots,Z_n$.  Given a
symmetric nonnegative loss function $d$ (acting over the space in which 
$\theta(P)$ lies), we define the \emph{minimax risk} over $\cP$ with respect to
$d$ to be   
\[
R_n = \inf_{\htheta} \, \sup_{P \in \cP} \, \E_P \big[ d(\theta(P), \htheta)
\big],  
\]
where the infimum is over all estimators \smash{$\htheta$}, and we use the
subscript $P$ on the expectation to refer to the fact that we are averaging over
the samples $Z_1,\dots,Z_n$ drawn from $P$, that are used to form
\smash{$\htheta$}.       

This may all look a little obscure. What does the class $\cP$ look like for some
typical problems? What about the functional $\theta(P)$, and the loss $d$?
Examples will help.    

\paragraph{Example: Gaussian mean estimation.}

As a simple parametric example, suppose that $\cP = \{ N(\theta,1) : \theta \in
\R\}$. For $P = N(\theta, 1)$, we can just take our functional to be $\theta(P)=
\theta$, the mean. Consider estimating the mean with the squared loss $d(a, b) 
= (a-b)^2$. The minimax risk is   
\[
R_n = \inf_{\htheta} \, \sup_\theta \, \E[ (\htheta - \theta)^2].
\]
It is implicit notationally that the expectation here is taken over i.i.d.\
samples $Z_1,\dots,Z_n \sim N(\theta,1)$, used to fit \smash{$\htheta$}.

For parametric models, where $\cP = \{P_\theta : \theta \in \Theta\}$ and
$\Theta \subseteq \R^d$, recall that under regularity conditions, the MLE has
risk $\lesssim \tr[I(\theta) ^{-1}] / n$ at $\theta$, where $I(\theta)$ is the
Fisher information matrix (and for typical models this will be of the order
$d/n$). Meanwhile, it can be shown that there is a local minimax lower
bound---local in the sense that the sup is taken over a neighborhood around
$\theta$---of the same order $\tr[I(\theta) ^{-1}] / n$. Thus the MLE is locally
minimax. In fact, it is more than this, because this statement can be made to be
uniform over all local neighborhoods around all $\theta \in \Theta$. This is due
to a general theory developed by H{\'a}jek and Le Cam, but we won't cover any of
this. We'll focus on nonparametric minimax theory (assuming you've seen
parametric minimax theory in previous courses).   

\paragraph{Example: nonparametric function estimation at a point, Random-X.} 

Let $Q$ be a fixed distribution on $[0,1]^d$ (e.g, the uniform distribution),
and let $Z_i = (x_i,y_i)$, $i=1,\dots,n$ be i.i.d.\ from $P$, with
\begin{equation}
\label{eq:random_x}
y_i = f(x_i) + \epsilon_i, \quad x_i \sim Q, \quad \epsilon_i \sim  
N(0,\sigma^2), \quad \text{and} \quad x_i \indep \epsilon_i,
\end{equation}
for some fixed $\sigma^2>0$. Let $\theta(P) = f$, which is an entire
function. Suppose that $\cP$ is the set of distributions $P$ of the form
\eqref{eq:random_x} for which $f \in \cF$, for some class of functions $\cF$ on 
$[0,1]^d$. To study function estimation at a single point---say, the origin---we
can take the loss to be \smash{$d(\hf, f) = (\hf(0) - f(0))^2$}. The minimax
risk is       
\begin{equation}
\label{eq:minimax_pointwise}
R_n = \inf_{\hf} \, \sup_{f \in \cF} \, \E\big[ (\hf(0) - f(0))^2 \big]. 
\end{equation}
The expectation is understood to be with respect to \eqref{eq:random_x}, which 
describes the samples used to fit \smash{$\hf$}. 

\paragraph{Example: nonparametric function estimation at a point, Fixed-X.} 

Similar to the last example, but now suppose that $y_i$, $i=1,\dots,n$ are
independent draws from $P$, with    
\begin{equation}
\label{eq:fixed_x}
y_i = f(x_i) + \epsilon_i, \quad x_i \;\, \text{fixed}, \quad \text{and} \quad
\epsilon_i \sim N(0,\sigma^2).
\end{equation}
We can still define the minimax risk as in \eqref{eq:minimax_pointwise}, where
now the expectation is understood to be with respect to \eqref{eq:fixed_x}. This 
requires some notational adjustment in the introductory paragraphs, because now
$y_i$, $i=1,\dots,n$ are independent but no longer i.i.d. (this will be true of
all Fixed-X models that we'll discuss henceforth). Similarly, we would need to  
adjust some of the techniques (Le Cam, Fano) that will be introduced below,
because as written they assume i.i.d.\ data. In several cases, these adjustments
will be straightforward and the minimax risk for the Random-X and Fixed-X models
will behave the same. However, interestingly, in other cases this will not be
true, and the minimax risk for the Random-X and Fixed-X models will be very
different. We'll discuss this at the end.

\paragraph{Example: nonparametric function estimation in population $L^2$ norm,
  Random-X.}  

As in our running example, under the Random-X model \eqref{eq:random_x},
consider the loss \smash{$d(\hf, f) = \|\hf - f\|_{L^2(Q)}^2$}, where recall $Q$
is the input distribution. This yields the minimax risk     
\begin{equation}
\label{eq:minimax_population}
R_n = \inf_{\hf} \, \sup_{f \in \cF} \, \E\Bigg[ \int (\hf(x) - f(x))^2 \, dQ(x)
\Bigg],   
\end{equation}
where the expectation is with respect to \eqref{eq:random_x}, which describes
the samples used to fit \smash{$\hf$}.  

% \paragraph{Example: nonparametric function estimation in empirical $L^2$ norm, 
%   Random-X.}  

% As in our running example, under the Random-X model \eqref{eq:random_x},
% consider the loss \smash{$d(\hf, f) = \|\hf - f\|_{L^2(Q)}^2$}, where $Q_n$
% denotes the empirical distribution of $x_i$, $i=1,\dots,n$. This yields the
% minimax risk     
% \begin{equation}
% \label{eq:minimax_empirical}
% R_n = \inf_{\hf} \, \sup_{f \in \cF} \, \E\Bigg[ \frac{1}{n} \sum_{i=1}^n
% (\hf(x_i) - f(x_i))^2 \Bigg], 
% \end{equation}
% where the expectation is with respect to \eqref{eq:random_x}, which describes
% the samples used to fit \smash{$\hf$}.  

\subsection{KL divergence}

\def\KL{\mathrm{KL}}

The \emph{Kullback-Leibler divergence} (KL) between two distributions $P,Q$,
having densities $p,q$, respectively, is 
\[
\KL(P,Q) = \int \log\bigg( \frac{dP}{dQ}(z) \bigg) \, dP(z)
= \int \log\bigg( \frac{p(z)}{q(z)} \bigg) p(z) \, dz.
\]
KL divergence will play a prominent role in a lot of the calculations that
follow. The following elementary fact will be useful for us. For Gaussians, $P = 
N(\theta, \sigma^2)$ and $Q = N(\mu, \sigma^2)$, we have  
\[
\KL(P, Q) = \frac{(\theta-\mu)^2}{2\sigma^2}.
\]

In general, $\KL(P,Q)$ is nonnegative and zero iff $P=Q$. This one of the
properties required of a distance (interpreting ``distance'' as being an
equivalent term to ``metric''). Yet KL divergence is not a distance, as it fails
each of the other two properties: it is not symmetric, nor does it satisfy the
triangle inequality.

Nonetheless, you'll sometimes hear people calling it ``KL distance''
anyway. There are many other distances on distributions (TV, $L^1$, Hellinger,
$\chi^2$, etc.) as well many relationships known between them, including
relationships to KL divergence. We do not review these here, but will simply
define other distances and use known relationships as they naturally arise in
what follows. See, e.g., Chapter 2.4 of \citet{tsybakov2009introduction} for a
more thorough treatment of distances on distributions.     

\subsection{Notation}

As in the nonparametric regression lectures, for sequences $a_n,b_n$, we will
write $a_n \lesssim b_n$ to mean $a_n = O(b_n)$, and we use $a_n \asymp b_n$ to
mean $a_n = O(b_n)$ and $b = O(a_n)$. We also use $a \wedge b = \min\{a,b\}$. 

\section{Standard reduction}

Typically we will not be interested in $R_n$ exactly, but only its dependence on
$n$. (We may also be interested in how it depends on auxiliary parameters that
define $\cP$. For example, in function estimation, if $\cF$ is a norm ball in
some function space, then we may also be interested in how $R_n$ scales with the
radius of this ball---and indeed, below, we'll track minimax rates as a function
of $n$ and the Lipschitz constant $L$ of the regression function.)  Of course,
if \smash{$\htheta$} is a particular estimator, then
\[
R_n \leq \sup_{P \in \cP} \, \E_P \big[ d(\theta(P), \htheta) \big], 
\]
so if the rate of convergence of \smash{$\htheta$} over the class of
distributions $\cP$ is (say) $n^{-w}$, then we learn $R_n\lesssim n^{-w}$.  

Finding a lower bound on $R_n$ will require a totally different technique, which
we will outline below. But if we can establish that $R_n \gtrsim n^{-w}$,
matching the upper bound in rate, then we conclude that $R_n \asymp n^{-w}$ and
we consider the case to be closed.

How do we find a lower bound? We reduce the problem to a hypothesis testing
problem. We do this because, in (certain simple) hypothesis testing problems,
it can be easier to reason about optimality. The general approach works like
this. Fix a finite set of distributions $S = \{P_1,\dots,P_N\} \subseteq
\cP$. Then  
\[
R_n = \inf_{\htheta} \, \sup_{P \in \cP} \, \E_P \big[ d(\theta(P), \htheta)
\big] \geq \inf_{\htheta} \, \max_{P_j \in S} \, E_j \big[ d(\theta_j, \htheta)
\big],  
\]
where we abbreviate where $\theta_j = \theta(P_j)$ and \smash{$E_j =
  \E_{P_j}$}. By Markov's inequality, for each $j$, and any $t>0$, 
\[
E_j \big[ d(\theta_j, \htheta) \big] \geq t P_j \big\{ d(\theta_j, \htheta) 
\geq t \big\}, 
\]
thus
\begin{equation}
\label{eq:finite_lb}
R_n \geq t \cdot \inf_{\htheta} \, \max_{P_j \in S} \, P_j \big\{ d(\theta_j,
\htheta) \geq t \big\}. 
\end{equation}
Any value of $t$ will give us a valid lower bound, but to find the ``right''
value of $t$, let's look at a calculation involving the minimum gap between
distinct $\theta_j$, $j=1,\dots,N$.

\paragraph{Minimum gap calculation.}

Define
\begin{equation}
\label{eq:min_gap}
s = \min_{j \not= k} \, d(\theta_j, \theta_k).
\end{equation}
Given an arbitrary estimator \smash{$\htheta$}, define 
\begin{equation}
\label{eq:psi_star}
\psi^* = \argmin_{j=1,\dots,N} \, d(\theta_j, \htheta).
\end{equation}

Let's assume that $d$ satisfies a quasi-triangle inequality, of the form 
\begin{equation}
\label{eq:quasi_triangle}
d(\theta, \theta') \leq C d(\theta,\theta'') + C d(\theta',\theta''), \quad
\text{for all $\theta,\theta',\theta''$},
\end{equation}
and a global constant $C>0$. For example, if $d$ is a metric, then it would 
satisfy \eqref{eq:quasi_triangle} with $C=1$, and if $d(\theta, \theta') =
\|\theta - \theta'\|_2^2$, then it would satisfy it with $C=2$. 

Now, if $\psi^* \not= j$, then letting $k = \psi^*$, observe that   
\begin{align*}
s &\leq d(\theta_j,\theta_k) \\
&\leq C d(\theta_j,\htheta) + C d(\theta_k,\htheta) \\
&\leq 2C d(\theta_j,\htheta).
\end{align*}
In the second line we use the quasi-triangle inequality, and in the third we
use \smash{$d(\theta_k, \htheta) \leq d(\theta_j, \htheta)$} (because $k 
=\psi^*$). Therefore we have shown that $\psi^* \not= j$ implies that  
\smash{$d(\theta_j, \htheta) \geq s/(2C)$}, and 
\begin{equation}
\label{eq:single_test}
P_j \bigg\{ d(\theta_j, \htheta) \geq \frac{s}{2C} \bigg\} \geq 
P_j (\psi^* \not= j).
\end{equation}

\paragraph{Back to minimax risk.}

Backing up, we have shown from \eqref{eq:finite_lb} and \eqref{eq:single_test},
plugging in $t = s/(2C)$, that  
\[
R_n \geq \frac{s}{2C} \cdot \inf_{\htheta} \, \max_{P_j \in S} \, 
P_j (\psi^*(\htheta) \not= j),
\]
where we write \smash{$\psi^* = \psi^*(\htheta)$} to emphasize its dependence on
\smash{$\htheta$}. But in fact we can go further. We can $\psi^*$ as defined in 
\eqref{eq:psi_star} as multiple hypothesis test: given access to
\smash{$\htheta$}, it tries to pick out which one of $\theta_j$ it thinks is
most likely. We can continue on lower bounding the right-hand side in the last
display by considering \emph{all} hypothesis tests that have access to the data
(on which the estimator \smash{$\htheta$} is fit). We'll summarize this in a
proposition for easy reference. 

\begin{proposition}
\label{prop:standard_reduction}
Let $S = \{P_1,\dots,P_N\} \subseteq \cP$ be any finite set, and $d$ be a
nonnegative symmetric loss satisfying the quasi-triangle inequality
\eqref{eq:quasi_triangle} with a constant $C>0$. Then 
\begin{equation}
\label{eq:standard_reduction}
R_n = \inf_{\htheta} \, \sup_{P \in \cP} \, \E_P \big[ d(\theta(P), \htheta)
\big] \geq \frac{s}{2C} \cdot \inf_\psi \, \max_{P_j \in S} \, P_j (\psi \not= 
j),  
\end{equation}
where $s$ is the minimum gap as in \eqref{eq:min_gap}, and the infimum is over
all maps $\psi$ from the data to $\{1,\dots,N\}$.  
\end{proposition}

This is called \emph{the standard reduction} for minimax lower bounds. Making
the best use of \eqref{eq:standard_reduction} (i.e., getting a tight lower
bound) requires carefully crafting $S = \{P_1,\dots,P_N\}$. If $S$ is too big
then $s$ will be small. But if $S$ is too small then \smash{$\max_{P_j \in S}
  P_j (\psi \not= j)$} will be small.    

\section{Le Cam's method}

\def\TV{\mathrm{TV}}

Le Cam's method is only a short hop away from the standard reduction. Consider
just two hypotheses $\theta_0 = \theta(P_0)$ and $\theta_1 = \theta(P_1)$, so
that $s = d(\theta_0,\theta_1)$. Let's also start with $n=1$ so we only have a
single observation. Then \eqref{eq:standard_reduction} tells us that
\[
R_n \geq \frac{s}{2C} \cdot \inf_\psi \, \max_{j=0,1} \, P_j(\psi \not= j). 
\]
Since a maximum is no smaller than an average,
\[
R_n \geq \frac{s}{4C} \cdot \inf_\psi \, \big[ P_0(\psi \not= 0) + P_1(\psi
\not= 1) \big]. 
\]
The reason that we switched from max testing risk to aggregate testing risk is 
that, for the latter, we know what optimality looks like: this is given by the
Neyman-Pearson test
\[
\psi_*(z) =
\begin{cases}
0 & \text{if $p_0(z) \geq p_1(z)$} \\
1 & \text{if $p_0(z) < p_1(z)$}
\end{cases}.
\]
We will use (without proof) the elementary yet critical fact that
$\inf_\psi [ P_0(\psi \not= 0) + P_1(\psi \not= 1)] = P_0(\psi_* \not= 0) +
P_1(\psi_* \not= 1)$. This is the essence of the Neyman-Pearson lemma.  

Now we compute
\begin{align*}
P_0(\psi_* \not= 0) + P_1(\psi_* \not= 1) 
&= \int_{p_1 > p_0} p_0(z) \, dz + \int_{p_0 \geq p_1} p_1(z) \, dz \\ 
&= \int_{p_1 > p_0} p_0(z) \wedge p_1(z) \, dz + \int_{p_0 \geq p_1} 
p_0(z)\wedge p_1(z) \, dz \\
&= \int p_0(z) \wedge p_1(z) \, dz.
\end{align*}
Thus we have shown that
\[
R_n \geq \frac{s}{2C} \frac{P_0(\psi_* \not= 0) + P_1(\psi_* \not= 1)}{2} 
= \frac{s}{4C} \int p_0(z) \wedge p_1(z) \, dz.
\]
Supposing we have $n$ observations, we replace $p_0$ and $p_1$ with 
\smash{$p_0^n(z) = \prod_{i=1}^n p_0(z_i)$} and \smash{$p_1^n(z) = \prod_{i=1}^n
  p_1(z_i)$}, and by the same arguments, we have
\begin{equation}
\label{eq:lecam_affinity}
R_n \geq \frac{s}{4C} \big[ P_0(\psi \not= 0) + P_1(\psi \not= 1) \big]  
= \frac{s}{4C} \int p_0^n(z) \wedge p_1^n(z) \, dz. 
\end{equation}
The integral on the right-hand side above is often called the \emph{affinity}
between $p_0^n$ and $p_1^n$. Using relationships between affinity, TV distance,
and KL divergence gives the set of results summarized in the next theorem. 

\begin{theorem}[Le Cam's lower bound]
\label{thm:lecam}
Let $P_0,P_1 \in \cP$, and let $d$ be a nonnegative symmetric loss satisfying
the quasi-triangle inequality \eqref{eq:quasi_triangle} with a constant
$C>0$. Then 
\begin{equation}
\label{eq:lecam_tv}
R_n \geq \frac{d(\theta_0, \theta_1)}{4C} \big[ 1 - \TV(P_0^n, P_1^n) \big], 
\end{equation}
where \smash{$\TV(P, Q) = \frac{1}{2} \int |p(z) - q(z)| \, dz$} denotes the
total variation distance between distributions $P,Q$ with densities
$p,q$. We also have the further lower bound
\begin{equation}
\label{eq:lecam_kl}
R_n \geq \frac{d(\theta_0, \theta_1)}{8C} e^{-n \KL(P_0, P_1)}. 
\end{equation}
\end{theorem}

The lower bounds in \eqref{eq:lecam_tv} and \eqref{eq:lecam_kl} simply come from
\eqref{eq:lecam_affinity}, combined with the following facts about affinity, TV
distance, and KL divergence of distributions $P,Q$ with densities $p,q$.   

\begin{itemize}
\item $\int p(z) \wedge q(z) \, dz = 1 - \TV(P,Q)$.
\item $\int p(z) \wedge q(z) \geq \frac{1}{2} e^{-\KL(P,Q)}$.
\item $\KL(P^n,Q^n) = n \cdot \KL(P,Q)$.
\end{itemize}

A useful corollary of Le Cam's KL bound \eqref{eq:lecam_kl} is the following. 

\begin{corollary}
\label{cor:lecam}
Under the same conditions on $d$ as in Theorem \ref{thm:lecam}, suppose there 
exists $P_0,P_1 \in \cP$ such that $\KL(P_0,P_1) \leq (\log 2)/n$. Then $R_n
\geq d(\theta_0, \theta_1)/(16 C)$.  
\end{corollary}

\subsection{Example: Lipschitz function estimation at a point, Random-X} 
\label{sec:lipschitz_pointwise}

We can demonstrate the utility of Le Cam's method by considering a Random-X
nonparametric regression model of the form \eqref{eq:random_x}. For simplicity,
let's take the input distribution to be uniform, $Q = \mathrm{Unif}([0,1]^d)$,
and just take $\sigma^2 = 1$. Consider $\cF = C^1(L; [0,1])^d$, the space of
functions that are $L$-Lipschitz continuous on $[0,1]^d$, and consider pointwise
risk at the origin, in squared loss, as in \eqref{eq:minimax_pointwise}.     

To be clear, in this context, $\theta_0 = f_0(0)$ and $\theta_1 = f_1(0)$, where 
$f_0,f_1$ are functions on $[0,1]^d$, and they are required to be Lipschitz in
order for $P_0,P_1 \in \cP$. Our loss is $d(\theta_0, \theta_1) = (f_0(0) -
f_1(0))^2$. Let's just fix $f_0 = 0$ (the zero function). Let $K$ be any
1-Lipschitz function supported on the unit $\ell_2$ ball $\{ x : \|x\|_2 \leq 1
\}$, such that $K(0) = 1$ and       
\[
0 < \int K(x)^2 \, dx < \infty.
\]
Then let $f_1(x) = Lh K(x/h)$, for a value $h > 0$ that we will specify
later. It is not hard to verify that $f_1$ is $L$-Lipschitz continuous.   
We compute
\allowdisplaybreaks
\begin{align*}
\KL(P_0,P_1) 
&= \int_{[0,1]^d} \int p_0(x,y) \log \bigg( \frac{p_0(x,y)}{p_1(x,y)} \bigg) \,
  dy \, dx \\  
&= \int_{[0,1]^d} \int p_0(y|x) \log \bigg( \frac{p_0(y|x)}{p_1(y|x)} \bigg) \,
  dy \, dx \\ 
&= \int_{[0,1]^d} \int \phi(y) \log \bigg( \frac{\phi(y)}{\phi(y - f_1(x))}
  \bigg) \, dy \, dx \\
&= \int_{[0,1]^d} \KL\big( N(0,1), N(f_1(x), 1) \big) \, dx \\
&= \frac{1}{2} \int_{[0,1]^d} f_1(x)^2 \, dx \\
&= \frac{L^2 h^2}{2} \int_{[0,1]^d} K(x/h)^2 \, dx \\
&\leq \frac{L^2 h^{2+d} \|K\|_2^2}{2}.
\end{align*}
In the second line, we use that $p_0(x) = p_1(x) = 1$ for all $x$; in the third,
we use $\phi$ for the standard normal density; in the fourth, we recognize the
inner integral as a KL divergence between $N(0,1)$ and $N(f_1(x),1)$; in the
fifth, we use the closed-form expression for the KL divergence between normals; 
and in the sixth and seventh, we recall the definition of $f_1$ and use variable 
substitution to compute the integral, denoting \smash{$\|K\|_2^2 = \int K(x)^2
  \, dx$}.   

Now let \smash{$h = ((2 \log 2)/(L^2 n \|K\|_2^2))^{1/(2+d)}$}. Then
$\KL(P_0,P_1) \leq (\log 2)/n$, so by Corollary \ref{cor:lecam} (where we note
that squared loss satisfies the quasi-triangle inequality
\eqref{eq:quasi_triangle} with $C=2$):     
\begin{align*}
\inf_{\hf} \, \sup_{f \in C^1(L; [0,1]^d)} \, \E\big[ (\hf(0) - f(0))^2 \big] 
&\geq \frac{f_1(0)^2}{32} \\ 
&= \frac{L^2 h^2}{32} \\
&\asymp L^{2d/(2+d)} n^{-2/(2+d)}. 
\end{align*}

Meanwhile, kNN regression or kernel smoothing can be shown to achieve the same
pointwise rate, which means we have found a tight lower bound.

\section{Fano's method}

When we move from a pointwise loss to an integrated loss, such as population or 
empirical $L^2$ loss, Le Cam's method---which only allows us to construct a pair  
of hypotheses that are hard to distinguish---is usually insufficient. 

Recall, however, that the standard reduction \eqref{eq:standard_reduction} was 
based on an arbitrarily large but finite set $S = \{P_1,\dots,P_N\} \subseteq
\cP$. Like we did in the derivation of Le Cam's method, we can use the fact that
a maximum is no smaller than an average, which gives 
\[
R_n \geq \frac{s}{2C} \cdot \inf_\psi \, \frac{1}{N} \sum_{j=1}^n P_j (\psi 
\not= j). 
\]
Now \emph{Fano's inequality}, a well-known result in information theory, tells
us that for any $\psi$,  
\[
\frac{1}{N} \sum_{j=1}^n P_j (\psi \not= j) \geq 1 - \frac{n\beta + \log 2}{\log
  N}, 
\]
where 
\begin{equation}
\label{eq:max_gap}
\beta = \max_{j \not= k} \, \KL(P_j, P_k).
\end{equation}
Putting this together gives the following result.

\begin{theorem}[Fano's lower bound]
\label{thm:fano}
Let $P_1,\dots,P_N \in \cP$, and let $d$ be a nonnegative symmetric loss
satisfying the quasi-triangle inequality \eqref{eq:quasi_triangle} with a
constant $C>0$. Then 
\begin{equation}
\label{eq:fano}
R_n \geq \frac{s}{2C} \bigg( 1 - \frac{n\beta + \log 2}{\log N} \bigg),
\end{equation}
where $s$ is the minimum $d$-gap as in \eqref{eq:min_gap}, and $\beta$ is the
maximum KL-gap as in \eqref{eq:max_gap}.
\end{theorem}

\begin{corollary}
\label{cor:fano}
Under the same conditions on $d$ as in Theorem \ref{thm:fano}, suppose there 
exists $P_1,\dots,P_N \in \cP$ such that $N \geq 4$ and $\beta \leq (\log
N)/(4n)$. Then $R_n \geq s/(8C)$.
\end{corollary}

There are many more methods for constructing lower bounds than just the Le Cam
and Fano methods. We won't cover these, but see, e.g., \citet{yu1997assouad,
  yang1999information}, as well as Chapter 2.7 of
\citet{tsybakov2009introduction}, for other techniques. 

\subsection{Varshamov-Gilbert lemma}

To use Fano's method, we need to construct a finite class of
distributions. Often we will use set of the form $\{P_\omega : \omega \in
\Omega\}$, where  
\[
\Omega = \{0,1\}^m = \Big\{ \omega = (\omega_1,\dots,\omega_m) : \omega_i \in
\{0,1\}, \, i=1,\dots,m \Big\},
\]
which is called a hypercube. There are $2^m$ elements in $\Omega$. For $\omega,
\nu\in \Omega$, their \emph{Hamming distance} is  
\[
H(\omega,\nu) = \sum_{i=1}^m 1\{ \omega_i \not= \nu_i \}.
\]
One ``problem'' with a hypercube, in terms of using it to index distributions
that we will construct, is that some pairs $P_\omega, P_\nu$ might be very close
together which will make the minimum $d$-gap, which recall is given in
\eqref{eq:min_gap}, too small. This will result in a poor lower bound.  

We can try to fix this problem by pruning the hypercube. That is, we will seek some
subset $\Omega' \subseteq \Omega$ having nearly the same number of elements
as $\Omega$, but where each pair $P_\omega, P_\nu$ is far apart in Hamming
distance, for $\omega, \nu \in \Omega'$ with $\omega \not= \nu$. The technique
for constructing such a pruned hypercube is given to us by what is known as the
\emph{Varshamov-Gilbert lemma}.   

\begin{lemma}[Varshamov-Gilbert]
\label{lem:vg}
Let $\Omega = \{0,1\}^m$, where $m \geq 8$. Then there exists a pruned hypercube
$\Omega' = \{ \omega^1,\dots,\omega^N \} \subseteq \Omega$ such that         
\begin{enumerate}
\item $N \geq 2^{m/8}$; and
\item $H(\omega^j,\omega^k) \geq m/8$ for each $j \not= k$.
\end{enumerate}
\end{lemma}

This is a standard result in information theory and its proof is somewhat
interesting because it involves randomization and Hoeffding's inequality, but we
won't cover it here. See, e.g., Chapter 2.6 in \citet{tsybakov2009introduction}.

\subsection{Example: Lipschitz function estimation in $L^2$ norm, Random-X}  
\label{sec:lipschitz_L2}

We now demonstrate the utility of Fano's method by consdering the same problem
setup as in Section \ref{sec:lipschitz_pointwise} but now with the squared $L^2$
loss defined with respect the uniform distribution $Q = \mathrm{Unif}([0,1]^d)$,
\[
d(\hf, f) = \|\hf - f\|_{L^2(Q)}^2 = \int_{[0,1]^d} (\hf(x) - f(x))^2 \, dx. 
\]
Note that in this context, each $\theta_j = f_j$, which is a particular
Lipschitz regression function, and the loss is \smash{$d(\theta_j, \theta_k) =
  \|f_j - f_k\|_{L^2(Q)}^2$}. We will define these regression functions below
using a two-step strategy. First we define locally-supported Lipschitz functions 
by translating a certain kernel function to be centered at points on a
grid. Then we use the Varshamov-Gilbert lemma to prune this set of regression
functions into a set whose minimum gap, as measured by the minimum of $d(f_j,
f_k)$, is large enough to yield a useful lower bound in Fano's method. 

As before, let $K$ be any 1-Lipschitz function supported on the unit ball $\{ x
: \|x\|_2 \leq 1 \}$, such that $K(0) = 1$ and  \smash{$0 < \int K(x)^2 \, dx
  < \infty$}. For an integer $r > 0$ to be specified later, define grid points  
\[
x_\alpha = \bigg( \frac{\alpha_1-1/2}{r}, \dots, \frac{\alpha_r-1/2}{r} \bigg)
\in [0,1]^d, \quad \text{for $\alpha \in [r]^d$}, 
\]
where we abbreviate $[r] = \{1,\dots,r\}$. Let $h=1/(2r)$ and define the functions 
\[
g_\alpha(x) = Lh K \bigg( \frac{x - x_\alpha}{h} \bigg), \quad \text{for $\alpha
  \in [r]^d$}.
\]
It is straightforward to check that each $g_\alpha$ is $L$-Lipschitz, and that
they have non-overlapping supports. Now just enumerate these functions as
$g_1,\dots,g_m$, for $m = r^d$, and define  
\[
f_\omega(x) = \sum_{i=1}^n \omega_i g_i(x), \quad \text{for $\omega \in
  \{0,1\}^m$}.  
\]
In other words, we construct each hypothesis $f_\omega$ by adding together some
subset of the locally-supported kernels $g_1,\dots,g_m$, this subset being
indexed by $\omega$. 

For $\omega,\nu \in \Omega$, note that by the non-overlapping supports property, 
\begin{align}
\nonumber
\int_{[0,1]^d} (f_\omega(x) - f_\nu(x))^2 \, dx
&= \int_{[0,1]^d} \bigg( \sum_{i=1}^m (\omega_i - \nu_i) g_i(x) \bigg)^2 \, dx
  \\    
\nonumber
&= H(\omega, \nu) \cdot L^2 h^2 \int_{[0,1]^d} K \bigg( \frac{x}{h} \bigg)^2 \,
  dx \\ 
\label{eq:omega_nu_d}
&= H(\omega, \nu) \cdot L^2 h^{2+d} \|K\|_2^2,
\end{align}
where $H(\omega, \nu)$ is the Hamming distance between $\omega,\nu$, and  
\smash{$\|K\|_2^2 = \int K(x)^2 \, dx$}. A similar calculation to that done in
the pointwise loss case shows that for the hypotheses $P_\omega,P_\nu$
corresponding to the regression functions $f_\omega,f_\nu$, respectively,  
\begin{align}
\nonumber
\KL(P_\omega,P_\nu) 
&= \frac{1}{2} \int_{[0,1]^d} (f_\omega(x) - f_\nu(x))^2 \, dx \\ 
\label{eq:omega_nu_kl}
&= H(\omega, \nu) \cdot L^2 h^{2+d} \|K\|_2^2 / 2,
\end{align}
with the calculation for the second line just following like that for
\eqref{eq:omega_nu_d}.  

At this point we apply the Varshamov-Gilbert lemma to produce a pruned hypercube
$\Omega' = \{\omega^1,\dots,\omega^N \} \subseteq \Omega = \{0,1\}^d$, with
cardinality $N \geq 2^{m/8}$, such that $H(\omega^j,\omega^k) \geq m/8$ for each
$j \not= k$. Then for each $j=1,\dots,N$, denote by $P_j$ the distribution
corresponding to the regression function $f_{\omega^j}$. Observe that, from 
\eqref{eq:omega_nu_d} and the lower bound on the Hamming distance over distinct
pairs in $\Omega'$,
\[
s = \min_{j \not= k} \, \| f_{\omega^j} - f_{\omega^k} \|_2^2 
\geq m L^2 h^{2+d} \|K\|_2^2 / 8 = c L^2 r^{-2}. 
\]
for a constant $c>0$. Meanwhile, from \eqref{eq:omega_nu_kl}, and the trivial
upper bound on the Hamming distance of $m$,
\[
\beta = \max_{j \not= k} \, \KL(P_j,P_k) \leq m L^2 h^{2+d} \|K\|_2^2 / 2 
= 4c L^2 r^{-2}.
\]
Finally, it is time to choose the grid side length $r$. We would like to have
$\beta \leq (\log N)/(4n)$ in order to be able to apply Corollary
\ref{cor:fano}. Recalling that $N \geq 2^{m/8}$, we have $\log N \geq (\log 2) m
/ 8 = (\log 2) r^d / 8$, so we want
\[
4c L^2 r^{-2} \leq (\log 2) r^d / (16n),
\]
which leads us to choose \smash{$r = \lceil c' (L^2 n)^{1/(2+d)} \rceil$} for
another constant $c'>0$. Corollary \ref{cor:fano} then tells us (using again
that squared loss satisfies the quasi-triangle inequality
\eqref{eq:quasi_triangle} with $C=2$) that 
\begin{align*}
\inf_{\hf} \, \sup_{f \in C^1(L; [0,1]^d)} \, \E \bigg[ \int_{[0,1]^d} (\hf(x) -
  f(x))^2 \, dx \bigg] 
&\geq \frac{s}{16} \\
&= \frac{cL^2 r^{-2}}{16} \\
&\asymp L^{2d/(2+d)} n^{-2/(2+d)}.
\end{align*}  

Recall, we know from our earlier nonparametric regression lecture that kNN
regression and kernel smoothing each achieve the above rate in squared $L^2$ 
norm, so we know that our lower bound is tight.  

\section{Fixed-X minimax theory?}

The Fixed-X minimax rate is not always the same as the Random-X rate. In some
cases, it the same; in other cases, it is different---and in fact, in particular 
cases, it is just about as different as it can get. The high-level conclusion is
that you have to be careful how you set up minimax estimation problems, in the
Fixed-X world.

Recall, in order to establish a minimax rate, we always require a matching upper 
and lower bound. The upper bounds from our previous lecture on kNN regression
and kernel smoothing were in the Random-X setting. The lower bounds constructed
thus far were also in the Random-X setting, and they matched, for the Lipschitz 
smoothness class.   

Below we walk through examples of Fixed-X minimax estimation in different
smoothness classes, in order to demonstrate how it can be similar in some cases
and different in others.

\subsection{Example: Lipschitz function estimation at a point, Fixed-X} 

To revisit the example from Section \ref{sec:lipschitz_pointwise}, suppose that
we change the problem setting from a Random-X to a Fixed-X model, i.e., now 
assuming \eqref{eq:fixed_x} instead of \eqref{eq:random_x}. Then $y_i$,
$i=1,\dots,n$ are independent but no longer i.i.d. Thankfully, very few changes
will be required to amend the arguments given earlier, with Le Cam's method in
the i.i.d.\ case. Careful inspection shows that we must only replace $P_j^n$, 
$j=0,1$ in \eqref{eq:lecam_affinity}, \eqref{eq:lecam_tv} with $P_{j1} \times
\cdots \times P_{jn}$, $j=0,1$, whose densities are \smash{$(P_{j1} \times
\cdots \times P_{jn})(z) = \prod_{i=1}^n p_{ji}(z_i)$}, $j=0,1$. After this
change, the lower bounds still hold. The KL bound \eqref{eq:lecam_kl} similarly
becomes 
\begin{equation}
\label{eq:lecam_kl_not_iid}
R_n \geq \frac{d(\theta_0,\theta_1)}{8C} e^{-\sum_{i=1}^n \KL(P_{0i}, P_{1i})}. 
\end{equation}
Using an analogous construction to that from Section
\ref{sec:lipschitz_pointwise}, we define $f_0=0$ and $f_1(x) = Lh K(x/h)$,
where $K$ is 1-Lipschitz, supported on the unit ball, with $K(0)=1$, and now
satisfies   
\[
\|K\|_n^2 = \frac{1}{n} \sum_{i=1}^n K(x_i)^2 = c,
\]
for some $0 < c < \infty$ that does not grow with $n$. Satisfying this last
requirement, which requires us to construct $K$ so that we have precise control
over its empirical norm, is easiest to do when $x_i$, $i=1,\dots,n$ are on a
regular lattice in $[0,1]^d$, which is a typical assumption in Fixed-X lower
bounds. 

Similar calculations to those in Section \ref{sec:lipschitz_pointwise} can be
used to show  
\[
\frac{1}{n} \sum_{i=1}^n \KL(P_{0i}, P_{1i}) = \frac{L^2 h^2}{2n}
\sum_{i=1}^n K(x_i/h) \lesssim L^2 h^{2+d}.
\]
From \eqref{eq:lecam_kl_not_iid}, we learn that if we set $h \asymp (L^2
n)^{-1/(2+d)}$, then we get
\begin{align*}
\inf_{\hf} \, \sup_{f \in C^1(L; [0,1]^d)} \, \E\big[ (\hf(0) - f(0))^2 \big] 
& \gtrsim f_1(0)^2 \\
&\asymp L^2 h^2 \\
&\asymp L^{2d/(2+d)} n^{-2/(2+d)}, 
\end{align*}
just as in the Random-X setting. 

Meanwhile, if we assume a ``grid design'', more precisely, we assume $N =
n^{1/d}$ is an integer and the input points form a regular grid on $[0,1]^d$
\[
\{x_1, \dots, x_n\} = [N]^d / N,
\]
then a matching upper bound can be constructed using kNN regression or kernel
smoothing. (For example, go back to own kNN analysis from last lecture, and
convince yourself it can be adapted.)  

% See also Chapter 1.6.1 of \citet{tsybakov2009introduction}. 

This means that $n^{-2/(2+d)}$ is still the minimax rate for pointwise loss over
the Lipschitz class $C^1(L; [0,1]^d$) in the Fixed-X grid design setting.

\subsection{Example: Lipschitz function estimation in $L^2$ norm, Fixed-X}  

To revisit the example from Section \ref{sec:lipschitz_L2}, suppose again that
we change the problem setting from Random-X to Fixed-X, i.e., assuming
\eqref{eq:fixed_x} instead of \eqref{eq:random_x}. Assume a grid design, as
above. Then a very similar calculation to that in Section \ref{sec:lipschitz_L2}
shows that in this Fixed-X model,    
\[
\inf_{\hf} \, \sup_{f \in C^1(L; [0,1]^d)} \, \E \bigg[ \int_{[0,1]^d} (\hf(x) -
f(x))^2 \, dx \bigg] \gtrsim n^{-2/(2+d)}, 
\]
just as in the Random-X setting. For details of the calculation, see, e.g.,
Chapter 2.6.1 of \citet{tsybakov2009introduction}.      

% See also Examples 2.7.3 and 2.7.6 in \citet{korostelev1993minimax}, the latter
% of which gives the d-dimensional rate.  

Meanwhile, a matching upper bound can be constructed using kNN or kernel
smoothing. (For example, go back to own kNN analysis from last lecture, and
convince yourself it can be adapted.)  

% See also Chapter 1.6.1 of \citet{tsybakov2009introduction}.  

This means that $n^{-2/(2+d)}$ is still the minimax rate for squared $L^2$ loss
over the Lipschitz class $C^1(L; [0,1]^d$) in the Fixed-X grid design
setting. (Note: remaining in Fixed-X, we can relax the grid design to a milder
condition on the inputs that requires them to sufficiently ``fill'' the domain
$[0,1]^d$, and the minimax rate in squared $L^2$ norm is still $n^{-2/(2+d)}$
over $C^1(L; [0,1]^d)$. See \citet{stone1982optimal}.) 

\subsection{Example: Sobolev function estimation, Random-X versus Fixed-X}   

Now let's take the example of nonparametric regression over the Sobolev class
$\cF = W^{s,2}(L; [0,1]^d)$, which we write to mean the set of functions $f$ on
$[0,1]^d$ that are $s$ times weakly differentiable with    
\[
\int_{[0,1]^d} \sum_{|\alpha| = s} [D^\alpha f(x)]^2 \, dx \leq L^2.
\]
As we have done thus far, assume for simplicity that $L$ does not grow with
$n$. To discuss minimax theory, we'll divide into cases.

\subsubsection{Random-X}

Consider the Random-X model \eqref{eq:random_x}. Assuming that the input
distribution is uniform on $[0,1]^d$ (or otherwise satisfies mild conditions),
one can show that 
\[
\inf_{\hf} \, \sup_{f \in W^{s,2}(L; [0,1]^d)} \, \E \bigg[ \int_{[0,1]^d}
(\hf(x) - f(x))^2 \, dx \bigg] \asymp n^{-2s/(2s+d)}.
\]
For the lower bound, we can appeal to facts about H{\"o}lder spaces. To see why
this is relevant, note that if $f \in C^s([0,1]^d)$ with H{\"o}lder constant
$M$, then $D^\alpha f$ is $M$-Lipschitz for all $|\alpha| = s-1$, which implies
(by Rademacher's theorem) that $f$ is $s$ times weakly differentiable and
$|D^\alpha f(x)| \leq M$ for all $|\alpha| = s$ and almost all $x$. Thus  
\[
\int_{[0,1]^d} \sum_{|\alpha| = s} [D^\alpha f(x)]^2 \leq N_{s,d} M^2,
\]
where $N_{s,d}$ is the number of multi-indices $\alpha \in \Z^d_+$ such that
$|\alpha| = s$ (its exact value is unimportant at the moment, but it is
\smash{$N_{s,d} = {s+d-1 \choose d-1}$}). In other words, we have shown $C^s(cL;
[0,1]^d) \subseteq W^{s,2}(L; [0,1]^d)$ for a constant $c$ that does not depend
on $n$. Therefore a lower bound on $C^s(cL; [0,1]^d)$ implies a lower bound on
$W^{s,2}(L; [0,1]^d)$. For the H{\"o}lder class $C^s(cL; [0,1]^d)$, we can 
construct a lower bound of order $n^{-2s/(2s+d)}$ for estimating H{\"o}lder
functions in squared $L^2$ norm, using arguments similar to what we did above
with Lipschitz functions (involving Varshamov-Gilbert and Fano).  You will
pursue this on the homework.

For the upper bound, a few different estimators achieve a squared $L^2$ norm 
error on the order $n^{-2s/(2s+d)}$ over the Sobolev class. For example,
spectral series estimators, and ``discretized'' versions of such estimators
which are based on the graph Laplacian. See \citet{green2023minimax}. To be 
clear there, are no restrictions on $s,d$ in any of this discussion. (The fact
that the minimax rate remains $n^{-2s/(2s+d)}$ in the regime $2s \leq d$ is  
a fairly remarkable feature of the Random-X setting. This will be more clear
once we cover the failures in the Fixed-X case, given shortly.)

\subsubsection{Fixed-X, $2s > d$}

Consider now the Fixed-X model \eqref{eq:fixed_x}, with a grid design. When $2s
> d$, one can show that
\[
\inf_{\hf} \, \sup_{f \in W^{s,2}(L; [0,1]^d)} \, \E \bigg[ \int_{[0,1]^d}
(\hf(x) - f(x))^2 \, dx \bigg] \asymp n^{-2s/(2s+d)},
\]
just as in the Random-X setting. The lower bound again comes from known results
in the H{\"o}lder class for Fixed-grid design. A matching upper bound can be
obtained using different techniques. One such example is given in
\citet{nussbaum1987nonparametric}, who considers regression onto a
tensor-product B-spline basis.   

\subsubsection{Fixed-X, $2s \leq d$}

In the Fixed-X model \eqref{eq:fixed_x}, with $2s \leq d$, the minimax rate over 
$W^{s,2}(L; [0,1]^d)$ a constant---meaning that the \emph{there is no estimator
  that is consistent over $W^{s,2}(L; [0,1]^d)$ in the sense of sup risk!} 

The underlying issue here is similar to what we encountered in the splines
lecture. Recall, when $2s \leq d$, we cannot generally talk about point
evaluation in a Sobolev space, in the sense that the point evaluation operator 
is not continuous. In a Fixed-X model, therefore, obtaining knowledge of
$f(x_i)$, $i=1,\dots,n$ doesn't help you reason about what $f$ looks like on the
rest of the domain $[0,1]^d$, for $f \in W^{s,2}(L; [0,1]^d)$. 

Indeed, we can make this idea explicit. Fix any small $\epsilon>0$, and let $f,g
\in W^{s,2}((1-\epsilon) L; [0,1]^d)$ be any two functions with \smash{$\|f - 
  g\|_{L^2([0,1]^d)} > 0$}. Define $\delta_i = g(x_i) - f(x_i)$, and 
\[
h_{i,N}(x) = \delta_i \cdot h(N(x-x_i)), \quad i=1,\dots,n.
\]
When $2s < d$, we can take $h$ to be the ``bump'' function used in the last
lecture: any smooth function that is unimodal about the origin, supported on  
the unit ball, with $h(0) = 1$. When $2m = d$, recall, we would need to modify
this construction somewhat, but the same key conclusions in what follows would
still hold. By the calculation from last time, for each $i = 1,\dots,n$, we
have: (i) \smash{$\|h_{i,N}\|_{W^{s,2}([0,1]^d)} \to 0$} as $N \to \infty$, and
(ii) \smash{$h_{i,N}(x_i) = \delta_i$} for each $N$. We now use these functions
to perturb the evaluations of $g$, defining: 
\[
\tilde{g}_N = g + \sum_{i=1}^n h_{i,N}.
\]
Note that the two properties above imply \smash{$\|\tilde{g}_N -
  g\|_{W^{s,2}([0,1]^d)} \to 0$} as $N \to \infty$ (so for large $N$, we will 
have \smash{$\tilde{g}_N \in W^{s,2}(L; [0,1]^d)$}), and
\smash{$\tilde{g}_N(x_i) = f(x_i)$} for each $i=1,\dots,n$ and each $N$. 

To summarize, we have constructed a sequence of functions, eventually lying in
the Sobolev space of interest, whose evaluations are equal to those of $f$:
\[
\max_{i=1,\dots,n} \, |f(x_i) - \tilde{g}_N(x_i)| = 0,
\]
yet is bounded away from $f$ in $L^2$ norm (as convergence in Sobolev norm
implies convergence in $L^2$ norm):  
\[
\lim_{N \to \infty} \, \| f - \tilde{g}_N \|_{L^2([0,1]^d)} = \|f -
g\|_{L^2([0,1]^d)} > 0. 
\]
It should therefore be clear that when $2s \leq d$ one cannot hope to estimate a
function in $W^{s,2}(L; [0,1]^d)$ in $L^2$ norm by observing its evaluations at
fixed points. 

We can formalize this using the Le Cam's two-point hypothesis method, adjusted
to the case of independent but not identically distributed data, as in
\eqref{eq:lecam_kl_not_iid}. We set the two distributions to be 
\begin{gather*}
P_{0i}: y_i \sim N(f(x_i), 1), \quad i = 1,\dots,n, \\
P_{1i} : y_i \sim N(\tilde{g}_N(x_i), 1), \quad i=1,\dots,n,
\end{gather*}
where in the current context $\theta_0 = f$, \smash{$\theta_1 = \tilde{g}_N$},
and \smash{$d(\theta_0,\theta_1) = \|f - \tilde{g}_N\|_{L^2([0,1]^d)} \geq
  1/2$}.  

For simplicity take $f = 0$ and $g = 1$ (constant functions). Then \smash{$\|f - 
  g\|_{L^2([0,1]^d)} = 1$} and we can take $N$ large enough so that \smash{$\|f
  - \tilde{g}_N\|_{L^2([0,1]^d)} \geq 1/2$}. Meanwhile, each $\KL(P_{0i},
P_{1i}) = 0$, since each \smash{$\tilde{g}_N(x_i) = f(x_i)$} by construction,
and therefore \eqref{eq:lecam_kl_not_iid} yields a minimax lower bound of 
\[
R_n \geq \frac{1}{32},
\]
where we have used again the fact that squared loss satisfies the quasi-triangle
inequality \eqref{eq:quasi_triangle} with $C=2$.

Reflecting back on the Random-X model, in light of the failures just discussed, 
it is actually pretty remarkable that estimation at the $L^2$ rate
$n^{-2s/(2s+d)}$ is possible when $2s \leq d$. In a sense, the randomness in the
inputs $x_i$, $i=1,\dots,n$ sort of finesses the problem of discontinuous point
evaluation.   

\bibliographystyle{plainnat}
\bibliography{../../common/ryantibs.bib}

\end{document}