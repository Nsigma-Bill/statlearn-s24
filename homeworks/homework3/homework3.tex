\documentclass{article}

\def\ParSkip{} 
\input{../../common/ryantibs}

\title{Homework 3 \\ \smallskip
\large Advanced Topics in Statistical Learning, Spring 2024 \\ \smallskip
Due Friday March 22 at 5pm}
\date{}

\begin{document}
\maketitle
\RaggedRight
\vspace{-50pt}

\section{Carath\'{e}odory's view on sparsity of lasso solutions [18 points]} 

In this exercise, we will prove the fact we cited in lecture about sparsity of
lasso solutions, by invoking Caratheodory's theorem. Let $Y \in \R^n$ be a
response vector, $X \in \R^{n \times d}$ be a predictor matrix, and consider
the lasso estimator defined by solving
\[
\minimize_\beta \; \frac{1}{2} \|Y - X\beta\|_2^2 + \lambda \|\beta\|_1,
\]
for a tuning parameter $\lambda > 0$.

\begin{enumerate}[label=(\alph*)]
\item Let \smash{$\hbeta$} be any solution to the lasso problem. Let
  \smash{$\hat\alpha = \hbeta / \|\hbeta\|_1$}. Prove that \smash{$X
    \hat\alpha$} lies in the convex hull of the vectors 
  \marginpar{\small [4 pts]}
  \[
  \{ \pm X_j \}_{j=1}^d.
  \]
  Note: here $X_j \in \R^n$ denotes the $j\th$ column of $X$.

\item Recall that Carath\'{e}odory's theorem states the following: given any set 
  $C \subseteq \R^k$, every element in its convex hull $\conv(C)$ can be 
  represented as a convex combination of $k+1$ elements of $C$. 

  Use this theorem and part (a) to prove that there exists a lasso solution
  \smash{$\tilde\beta$} with at most $n+1$ nonzero coefficients.  
  \marginpar{\small [4 pts]}

  Hint: start with a generic solution \smash{$\hbeta$}, and use
  Carath\'{e}odory's theorem to construct a coefficient vector
  \smash{$\tilde\beta$} such that (i) the fit is the same, \smash{$X\tilde\beta
    = X\hbeta$}; (ii) the penalty is at worst the same,
  \smash{$\|\tilde\beta\|_1 \leq \|\hbeta\|_1$}; and (iii)
  \smash{$X\tilde\beta$} is a nonnegative linear combination of at most $n+1$ of
  $\pm X_j$, $j=1,\dots,d$.  
 
\item Now, assuming $\lambda>0$, use the subgradient optimality condition for 
  the lasso problem to prove that the fit \smash{$X\tilde\beta$} from part (b)
  is supported on a subset of 
 \marginpar{\small [6 pts]}
  \[
  \{ \pm X_j \}_{j=1}^d
  \]
  that has affine dimension at most $n-1$.

  Hint: this is similar to the proof of Proposition 1 in the lasso lecture
  notes. Assume that \smash{$X\tilde\beta$} is a nonnegative combination of 
  exactly $n+1$ of $\pm X_j$, $j=1,\dots,d$. Then one of these $n+1$ vectors, 
  denote it by $s_i X_i$ (where \smash{$s_i = \sign(\tilde\beta_i)$}) can be
  written as a linear combination of the others. Take an inner product with the
  lasso residual and use the subgradient optimality condition for the lasso to
  prove that the coefficients in this linear combination must sum to 1, and
  therefore, $s_i X_i$ is actually an affine combination of the others. Notice
  that this shows the affine span of the $n+1$ vectors in question is
  $(n-1)$-dimensional.     

\item A refinement of Carath\'{e}odory's is as follows: given a set $C \subseteq
  \R^k$, every element in its convex hull $\conv(C)$ can be represented as a
  convex combination of $r+1$ elements of $C$, where $r$ is the affine dimension
  of $\conv(C)$. 

 Use this theorem and part (c) to prove that there exists a lasso solution
  \smash{$\check\beta$} with at most $n$ nonzero coefficients.  
  \marginpar{\small [4 pts]}
\end{enumerate}

\section{Variance of least squares in nonlinear feature models [30 points]} 

\def\asto{\overset{\mathrm{as}}{\to}}
\def\hSigma{\hat\Sigma}

In this exercise, we will examine the variance of least squares (in the
underparametrized regime) and min-norm least squares (in the overparametrized 
regime) in nonlinear feature models. Recall for a response vector $Y \in \R^n$
and feature matrix $X \in \R^{n \times d}$, the min-norm least squares estimator 
\smash{$\hbeta = (X^\T X/n)^+ X^\T Y/n$} has a variance component of its
out-of-sample prediction risk (conditional on $X$) given by:
\begin{equation}
\label{eq:ridgeless_var}
V_X(\hbeta) = \frac{\sigma^2}{n} \tr (\hSigma^+ \Sigma).
\end{equation}
Here \smash{$\hSigma = X^\T X/n$}, and $\Sigma = \Cov(x_i)$, for an arbitrary
row $x_i$ of $X$ (the rows all have the same distribution). Also, $\sigma^2 =
\Var[y_i | x_i]$ is the noise variance. In lecture, we studied a linear feature 
model of the form    
\begin{equation}
\label{eq:linear_features}
X = Z \Sigma^{1/2},
\end{equation}
for a covariance matrix $\Sigma \in \R^{d \times d}$ and a random matrix $Z \in
\R^{n \times d}$ that has i.i.d.\ entries with mean zero and unit variance. When
$\Sigma = I$, which we will assume throughout this homework problem, recall that
we proved that the variance \eqref{eq:ridgeless_var} satisfies, under standard 
random matrix theory conditions, as $n,d \to \infty$ and $d/n \to \gamma \in 
(0,\infty)$, 
\begin{equation}
\label{eq:ridgeless_var_iso_limit}
V_X(\hbeta) \asto 
\begin{cases}
\sigma^2 \frac{\gamma}{1-\gamma} & \text{for $\gamma < 1$} \\ 
\sigma^2 \frac{1}{\gamma-1} & \text{for $\gamma > 1$}.
\end{cases}
\end{equation}
(The result for $\gamma < 1$ actually holds regardless of $\Sigma$.) Instead, we
can consider a nonlinear feature model of the form 
\begin{equation}
\label{eq:nonlinear_features}
X = \varphi(Z \Gamma^{1/2} W^\T),
\end{equation}
for a covariance matrix $\Gamma \in \R^{k \times k}$, and a random matrix $Z \in
\R^{n \times k}$ as before (except with $k$ in place of $d$). Moreover, now $W
\in \R^{d \times k}$ is a matrix of i.i.d.\ $N(0,1/k)$ entries, and $\varphi :
\R \to \R$ is a nonlinear function---called the activation function in a neural
network context---that we interpret to act elementwise on its input. 

There turns to be an uncanny connection between the asymptotic variance in
linear and nonlinear feature models, which will you uncover via simulation in
this homework problem. Attach your code as an appendix to this homework.  

For parts (a)--(d) below, consider isotropic features, so that $\Sigma = I$ in
\eqref{eq:ridgeless_var} and \eqref{eq:linear_features}, and $\Gamma = I$ in
\eqref{eq:nonlinear_features}.  

\begin{enumerate}[label=(\alph*)]
\item Fix $n=200$, and let $d=[\gamma n]$ over a wide range of values for
  $\gamma$ (make sure your range covers both $\gamma<1$ and $\gamma>1$). 
  each $n,d$, draw $X$ from the linear feature model \eqref{eq:linear_features}
  and your choice of distribution for the entries of $Z$. Compute the
  finite-sample variance \eqref{eq:ridgeless_var}, and plot it, as a function of
  $\gamma$, on top of the asymptotic variance curve
  \eqref{eq:ridgeless_var_iso_limit}. To get a general idea of what this should
  look like, refer back to Figure 2 in the overparametrization lecture notes.      
  \marginpar{\small [6 pts]}

\item For the same values of $n,d$, and $k=100$, draw $X$ from the nonlinear
  feature model \eqref{eq:nonlinear_features}, for three different choices of
  $\varphi$:   
  \begin{enumerate}[label=\roman*.]
  \item $\varphi(x) = a_1\tanh(x)$;
  \item $\varphi(x) = a_2(x_+-b_2)$;
  \item $\varphi(x) = a_3(|x|-b_3)$.
  \end{enumerate}
  Here $a_1,a_2,b_2,a_3,b_3$ are constants that you must choose to meet the 
  standardization conditions $\E[\varphi(G)]=0$ and $\E[\varphi(G)^2]=1$, for 
  $G \sim N(0,1)$. Produce a plot just as in part (a), with the finite-sample
  variances for choice of each activation function plotted in a different color,
  on top of the asymptotic variance curve \eqref{eq:ridgeless_var_iso_limit} for  
  the linear model case. Comment on what you find: do the nonlinear
  finite-sample variances lie close to the asymptotic variance for the linear
  model case?  
  \marginpar{\small [18 pts]}

\item Now use a linear activation function $\phi(x) = ax-b$, and create a plot
  as in part (b) with the same settings (same values of $n,d,k$, and so
  on). What behavior do the finite-sample variances have as a function of 
  $\gamma$? Is this surprising to you? Explain why what you are seeing is
  happening. 
 \marginpar{\small [6 pts]}

\item As a bonus, in light of part (c), elaborate on why the results in part (b)
  are remarkable. 

\item As another (large) bonus, rerun the analysis in this entire problem but
  with a non-isotropic covariance $\Sigma$ in \eqref{eq:linear_features}, and
  $\Gamma$ in \eqref{eq:nonlinear_features}. Extra bonus points if you properly
  recompute the asymptotic variance curves. 
\end{enumerate}

\section{The implicit regularization of gradient flow [25 points]} 

\def\gf{\mathrm{gf}}
\def\ridge{\mathrm{ridge}}

We will study gradient flow, as a continuous-time limit of the gradient descent
path, in least squares regression. To build up motivation, consider gradient
descent applied to the least squares regression problem 
\[
\minimize_\beta \; \frac{1}{2n} \|Y - X\beta\|_2^2,
\]
for a response vector $Y \in \R^n$ and predictor matrix $X \in \R^{n \times
  d}$. For a given fixed step size $\epsilon>0$, and for an initialization
\smash{$\beta^{(0)}=0$}, gradient descent repeats the iterations:  
\[
\beta^{(k)} = \beta^{(k-1)} + \epsilon \cdot \frac{X^\T}{n} 
(Y - X \beta^{(k-1)}), 
\]
for $k=1,2,3,\ldots$. Rearranging gives
\[
\frac{\beta^{(k)} - \beta^{(k-1)}}{\epsilon} = \frac{X^\T}{n} (Y - X
\beta^{(k-1)}),  
\]
and letting $k \to \infty$ and $\epsilon \to 0$, in such a way that $k\epsilon =
t$, we get a continuous-time ordinary differential equation 
\begin{equation}
\label{eq:gradient_flow}
\dot\beta(t) = \frac{X^\T}{n} (Y - X \beta(t)),
\end{equation}
over time $t>0$, subject to an initial condition $\beta(0) = 0$. We refer to the
solution as the gradient flow path for least squares regression. 

\begin{enumerate}[label=(\alph*)]
\item Prove that the gradient glow path, the solution in
  \eqref{eq:gradient_flow}, is 
 \marginpar{\small [4 pts]}
  \[
  \hbeta^\gf(t) = (X^\T X)^+ (I - \exp(-t X^\T X/n)) X^\T Y.
  \]
  Here $A^+$ is the Moore-Penrose generalized inverse of a matrix $A$, and
  $\exp(A) = I + A + A^2/2! + A^3/3! + \cdots$ is the matrix exponential of
  $A$. Note: you may use whatever properies of the matrix exponential that you
  want, as long as you clearly state them. (Also, you do not have to prove
  uniqueness of the solution in \eqref{eq:gradient_flow}, you just have to plug
  in the above expression and show that it solves \eqref{eq:gradient_flow}.) 

\item Let \smash{$X = \sqrt{n} U S^{1/2} V^\T$} be a singular value
  decomposition, so that $X^\T X / n = V S V^\T$ is an
  eigendecomposition. Letting $u_i$, $i=1,\dots,p$ denote the columns of $U$, 
  and $s_i$, $i=1,\dots,p$ the diagonal entries of $S$, prove that the vector of
  in-sample predictions from gradient flow are   
 \marginpar{\small [4 pts]}
  \[
  X \hbeta^\gf(t) = \sum_{i=1}^p (1 - \exp(-ts_i)) u_i u_i^\T Y.
  \]
 
\item Recall, the vector of in-sample predictions from ridge regression with
  tuning parameter $\lambda>0$,
  \[
  \minimize_\beta \; \frac{1}{n} \|Y - X\beta\|_2^2 + \lambda \|\beta\|_2^2,  
  \]
 can be written as 
  \[
  X \hbeta^\ridge(\lambda) = \sum_{i=1}^p \frac{s_i}{s_i + \lambda} u_i u_i^\T
  Y. 
  \]
  Thus both gradient flow and ridge perform a shrunken regression, by shrinking
  the eigenvalues of the empirical covariance matrix, but they use different
  underlying shrinkage maps, respectively:
  \begin{align*}
  g^\gf(s,t) &= 1 - \exp(-ts), \\
  g^\ridge(s,\lambda) &= \frac{s}{s + \lambda}.
  \end{align*}
  Plot \smash{$g^\ridge$} as a heatmap with $s$ on the x-axis, and $\lambda$ on
  the y-axis. Then, using the parametrization $t = 1/\lambda$, plot 
  \smash{$g^\gf$} as a heatmap, again over $s$ (x-axis) and $\lambda$
  (y-axis). Do you notice a similarity?     
  \marginpar{\small [4 pts]}

\item Under the model 
  \begin{gather*}
  Y = X\beta_0 + \epsilon, \\ \text{where $\E[\epsilon] = 0$, $\Cov(\epsilon) =
    \sigma^2 I$}, 
  \end{gather*}
  and with $X$ treated as fixed, prove that the estimation risk of gradient flow
  is  
  \marginpar{\small [8 pts]}
  \[
  \E \|\hbeta^\gf(t) - \beta_0\|_2^2 = \sum_{i=1}^p \bigg(|v_i^\T \beta_0|^2 
  \exp(-2 t s_i) +  \frac{\sigma^2}{n} \frac{(1 - \exp(-t s_i))^2}{s_i} \bigg), 
  \]
  where recall $X^\T X/n = V^\T S V$ is an eigendecomposition, and we use $s_i$,
  $i=1,\dots,p$ for the diagonal entries of $S$, and $v_i$, $i=1,\dots,p$ for
  the columns of $V$. 

\item Prove that, under the model from part (d), 
  \marginpar{\small [5 pts]}
  \[
  \E \|\hbeta^\gf(1/\lambda) - \beta_0\|_2^2 \leq 1.6862 \cdot \E
  \|\hbeta^\ridge(\lambda) - \beta_0\|_2^2,
  \]
  for any $\lambda > 0$. In words, the estimation risk of gradient flow at time
  $t = 1/\lambda$ is no more than 1.6862 times that of ridge at regularization 
  parameter $\lambda$.

  Hint: you may use the fact that the estimation risk of ridge is 
  \[
  \E \|\hbeta^\ridge(\lambda) - \beta_0\|_2^2 = \sum_{i=1}^p \bigg( |v_i^\T 
  \beta_0|^2 \frac{\lambda^2}{(s_i + \lambda)^2} + \frac{\sigma^2}{n}
  \frac{s_i}{(s_i + \lambda)^2} \bigg). 
  \]
  Bonus: derive this. Another hint: you may use the fact that for all $x \geq
  0$, it  
  holds that 
  \begin{enumerate}[label=(\roman*)]
  \item $e^{-x} \leq 1/(1+x)$, and
  \item $1-e^{-x} \leq 1.2985 \cdot x / (1+x)$.
  \end{enumerate}
  Bonus: prove these facts.
\end{enumerate}

\end{document}